\chapter{Antecedentes}

\section{Introducción}
En este apartado se analizan brevemente herramientas existentes que, o bien son similares o bien tienen objetivos parecidos a \emph{PushNews}. Se han seleccionado herramientas de comunicación corporativa como boletines informativos o webs corporativas y también de marketing digital como las que se utilizan para gestionar redes sociales.

\section{Newsletter o boletín informativo}
Una newsletter es una publicación digital periódica que se distribuye a través de correo electrónico. Generalmente recopila los artículos o comunicados publicados desde el último envío. Los receptores de una newsletter son personas que han mostrado previamente interés en el tema y se han suscrito a la publicación para recibirla en su bandeja de correo electrónico. La newsletter sigue siendo a día de hoy una herramienta muy importante en la comunicación corporativa por varias razones según Wikipedia \cite{wiki_boletininformativo}:
\subsection*{Ventajas}
\begin{enumerate}
    \item Es un canal que, bien usado, permite construir una relación de confianza con el suscriptor que puede ser la base de futuras ventas.
    \item La calidad como lectores/clientes de las personas suscritas suele ser superior a los lectores esporádicos que encuentran llegan al sitio web, por ejemplo, a través de los buscadores.
    \item Porque es un activo que el autor de la lista tiene bajo su total control, a diferencia, por ejemplo, de los seguidores en redes sociales como Facebook o Twitter.
\end{enumerate}

\subsection*{Desventajas}
\begin{enumerate}
    \item Puede ser difícil conseguir que el que el usuario dé su consentimiento para recibir la newsletter.
    \item A menudo se considera spam.
    \item Es fácil que el usuario no preste atención a la newsletter entre todo el contenido de su bandeja de entrada.
\end{enumerate}

\section{Web corporativa}
Una web corporativa es un tipo de sitio web orientado a dar a conocer información sobre una empresa o institución. Aunque ya no supone un elemento diferenciador debido a que la mayoría de las organizaciones tienen una, resulta fundamental ya que proporciona la organización presencia en internet sin depender de terceros. Según The Company Warehouse \cite{web_corporativa}, las principales ventajas y desventajas de las webs corporativas son:
\subsection*{Ventajas}
\begin{enumerate}
    \item Potencialmente, llegan a una audiencia muy amplia.
    \item La información sobre la organización está siempre disponible y es fácilmente accesible.
    \item Grandes posibilidades de publicidad y marketing viral.
\end{enumerate}
\subsection*{Desventajas}
\begin{enumerate}
    \item Necesidad de mantener la información actualizada para no perder la confianza del público.
    \item Llegar a al público adecuado. Debido a la gran competencia existente, es muy difícil aparecer arriba en las búsquedas.
    \item Necesidad de mantener la web libre de errores y caídas de servicio.
\end{enumerate}

\section{Redes sociales}
Debido a su gran impacto social, las redes sociales se han convertido en un lugar obligado para las organizaciones donde ejercer su presencia digital gracias a sus posibilidades de segmentación, que permite llegar al público objetivo fácilmente. Según \cite{nadaraja2013socialmediamarketing}, estas son las principales ventajas y desventajas de las redes sociales:

\subsection*{Ventajas}
\begin{enumerate}
    \item Las barreras económicas son prácticamente inexistentes ya que la mayoría de redes sociales son de acceso gratuito. 
    \item La audiencia que desea recibir la información sigue las redes sociales de la organización y la comparten, ayudando a difundirla sin costes.
    \item La audiencia en las redes sociales está fuertemente segmentada, lo que permite llegar fácilmente a las personas realmente interesadas.
\end{enumerate}

\subsection*{Desventajas}
\begin{enumerate}
    \item Requieren mucho trabajo: debido a la naturaleza bidireccional de las interacciones en redes, atender a todas en cada una de las redes sociales supone una gan inversión de tiempo.
    \item Confianza, privacidad: al utilizar una plataforma de terceros, es obligatorio aceptar y someterse a las políticas de seguridad y privacidad de dicha plataforma.
    \item Respuesta negativa: dado que cualquier usuario de la red social puede generar contenido asociado al de una organización, esta se expone a recibir una respuesta negativa de la audiencia que tiene un gran impacto sobre su credibilidad.
\end{enumerate}

\subsection{Herramientas de gestión de redes sociales}
Existen soluciones que ayudan a gestionar las diferentes redes sociales con herramientas que facilitan las labores comunes del trabajo en redes sociales \cite{herramientas-social-media-1} \cite{herramientas-social-media-2} paliando algunas de las desventajas citadas anteriormente. Ejemplos de estas herramientas son Hootsuite, Keyhole, Narrow, Buffer. 

Las funcionalidades principales que aporta este tipo de herramientas son:
\begin{itemize}
    \item Publicar comunicados en varias redes a la vez.
    \item Planificar la publicación en el tiempo.
    \item Facilitar la gestión de las interacciones.
    \item Medir el impacto de los comunicados publicados.
\end{itemize}