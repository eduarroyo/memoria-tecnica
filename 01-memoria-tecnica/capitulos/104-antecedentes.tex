\chapter{Antecedentes}

\section{Introducción}
En este apartado se analizan brevemente herramientas existentes que, o bien son similares o bien tienen objetivos parecidos a \emph{PushNews}. Se han seleccionado herramientas de comunicación corporativa como boletines informativos o webs corporativas y también de marketing digital como las que se utilizan para gestionar redes sociales.

\section{Newsletter o boletín informativo}
Una newsletter es una publicación digital periódica que se distribuye a través de correo electrónico. Generalmente recopila los artículos o comunicados publicados desde el último envío. Los receptores de una newsletter son personas que han mostrado previamente interés en el tema y se han suscrito a la publicación para recibirla en su bandeja de correo electrónico. La newsletter sigue siendo a día de hoy una herramienta muy importante en la comunicación corporativa por varias razones según Wikipedia \cite{wiki_boletininformativo}:
\begin{enumerate}
    \item Es un canal que, bien usado, permite construir una relación de confianza con el suscriptor que puede ser la base de futuras ventas.
    \item La calidad como lectores/clientes de las personas suscritas suele ser superior a los lectores esporádicos que encuentran llegan al sitio web, por ejemplo, a través de los buscadores.
    \item Porque es un activo que el autor de la lista tiene bajo su total control, a diferencia, por ejemplo, de los seguidores en redes sociales como Facebook o Twitter.
\end{enumerate}

Por otra parte, las newletters presentan algunas desventajas como las siguientes:
\begin{enumerate}
    \item Puede ser difícil conseguir que el que el usuario dé su consentimiento para recibir la newsletter.
    \item A menudo se considera spam.
    \item Es fácil que el usuario no preste atención a la newsletter entre todo el contenido de su bandeja de entrada.
\end{enumerate}

\section{Web corporativa}
Una web corporativa es un tipo de sitio web orientado a dar a conocer información sobre una empresa o institución. Esta herramienta proporciona a una organización presencia en internet sin depender de terceros. 

\section{Redes sociales}

\subsection{Herramientas de gestión de redes sociales}
\cite{herramientas-social-media}

\subsection{Bots}