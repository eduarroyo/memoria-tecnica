\chapter{Análisis funcional}
En este capítulo se detalla el funcionamiento del sistema. Primero se identificará a los actores que intervienen en el sistema, indicando los objetivos de cada uno. A continuación se presentarán los escenarios empleando casos de uso. Por último se detallarán las interacciones entre los actores y los demás elementos del sistema en los principales escenarios mediante diagramas de secuencia.

%%%%%%%%%%%%%%%%%%%%%%%%%%%%%%%%%%%%%%%%%%%%%%%%%%%%%%%%%%%%%%%%%%%%%%%%%%%%%%%%%%%%%%%%%%%%%%%%%%%
%%%%%%%%%%%%%%%%%%%%%%%%%%%%%%%%%%%%%%%%%%%%%%%%%%%%%%%%%%%%%%%%%%%%%%%%%%%%%%%%%%%%%%%%%%%%%%%%%%%
%%%%%%%%%%%%%%%%%%%%%%%%%%%%%%%%%%%%%%%%%%%%%%%%%%%%%%%%%%%%%%%%%%%%%%%%%%%%%%%%%%%%%%%%%%%%%%%%%%%
%%%%%%%%%%%%%%%%%%%%%%%%%%%%%%%%%%%%%%%%%%%%%%%%%%%%%%%%%%%%%%%%%%%%%%%%%%%%%%%%%%%%%%%%%%%%%%%%%%%
%%%%%%%%%%%%%%%%%%%%%%%%%%%%%%%%%%%%%%%%%%%%%%%%%%%%%%%%%%%%%%%%%%%%%%%%%%%%%%%%%%%%%%%%%%%%%%%%%%%
\section{Identificación de actores}
Como ya se introdujo en \ref{objetivos-especificos-gestor-contenidos}, en \emph{PushNews} existen tres tipos de actores o tres perfiles. A continuación se profundizará en los objetivos de cada uno de ellos.

\subsection{Lector}
Cualquier usuario que, incluso sin estar previamente registrado y sin haberse identificado en el sistema, consulte una comunicación. Los principales objetivos de los lectores son:
\begin{enumerate}
    \item Visualizar las comunicaciones publicadas.
    \item Obtener los recursos adjuntos a las comunicaciones (imágenes, documentos\dots).
    \item Saber cuando una nueva comunicación es publicada.
\end{enumerate}

\subsection{Editor}
Generalmente, los editores pertenecerán a la organizaciones de los clientes y su misión es publicar y mantener las comunicaciones así como consultar las métricas. Sus objetivos más importantes son:
\begin{enumerate}
    \item Publicar comunicaciones
    \item Gestión o mantenimiento de las comunicaciones (modificación, eliminación\dots)
    \item Gestión de las categorías de comunicaciones de la aplicación.
    \item Gestión de otros datos de las aplicaciones como lugares de interés, teléfonos, etc.
    \item Ver las métricas de accesos a las comunicaciones
\end{enumerate}

\subsection{Administrador}
Los administradores se encargan del mantenimiento del sistema desde la organización de \emph{PushNews}. Entre sus objetivos principales encontramos:
\begin{enumerate}
    \item Mantenimiento general del servicio.
    \item Administrar las aplicaciones (altas, bajas, etc\dots).
    \item Administrar las características asociadas a las aplicaciones.
    \item Administración de los usuarios (editores) de las aplicaciones. 
    \item Realizar labores de editor de cualquier aplicación.
\end{enumerate}

%%%%%%%%%%%%%%%%%%%%%%%%%%%%%%%%%%%%%%%%%%%%%%%%%%%%%%%%%%%%%%%%%%%%%%%%%%%%%%%%%%%%%%%%%%%%%%%%%%%
%%%%%%%%%%%%%%%%%%%%%%%%%%%%%%%%%%%%%%%%%%%%%%%%%%%%%%%%%%%%%%%%%%%%%%%%%%%%%%%%%%%%%%%%%%%%%%%%%%%
%%%%%%%%%%%%%%%%%%%%%%%%%%%%%%%%%%%%%%%%%%%%%%%%%%%%%%%%%%%%%%%%%%%%%%%%%%%%%%%%%%%%%%%%%%%%%%%%%%%
%%%%%%%%%%%%%%%%%%%%%%%%%%%%%%%%%%%%%%%%%%%%%%%%%%%%%%%%%%%%%%%%%%%%%%%%%%%%%%%%%%%%%%%%%%%%%%%%%%%
%%%%%%%%%%%%%%%%%%%%%%%%%%%%%%%%%%%%%%%%%%%%%%%%%%%%%%%%%%%%%%%%%%%%%%%%%%%%%%%%%%%%%%%%%%%%%%%%%%%
\section{Análisis de casos de uso}
Según Larman, un caso de uso es ``una colección de escenarios con éxito y fallo relacionados, que describe a los actores utilizando un sistema para satisfacer un objetivo.'' \cite{Larman2004}. En este apartado se detallarán los principales casos de uso del sistema.

%%%%%%%%%%%%%%%%%%%%%%%%%%%%%%%%%%%%%%%%%%%%%%%%%%%%%%%%%%%%%%%%%%%%%%%%%%%%%%%%%%%%%%%%%%%%%%%%%%%
%%%%%%%%%%%%%%%%%%%%%%%%%%%%%%%%%%%%%%%%%%%%%%%%%%%%%%%%%%%%%%%%%%%%%%%%%%%%%%%%%%%%%%%%%%%%%%%%%%%
%%%%%%%%%%%%%%%%%%%%%%%%%%%%%%%%%%%%%%%%%%%%%%%%%%%%%%%%%%%%%%%%%%%%%%%%%%%%%%%%%%%%%%%%%%%%%%%%%%%
%%%%%%%%%%%%%%%%%%%%%%%%%%%%%%%%%%%%%%%%%%%%%%%%%%%%%%%%%%%%%%%%%%%%%%%%%%%%%%%%%%%%%%%%%%%%%%%%%%%
%%%%%%%%%%%%%%%%%%%%%%%%%%%%%%%%%%%%%%%%%%%%%%%%%%%%%%%%%%%%%%%%%%%%%%%%%%%%%%%%%%%%%%%%%%%%%%%%%%%
\section{Diagramas de secuencia}