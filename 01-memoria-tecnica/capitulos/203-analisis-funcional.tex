\chapter{Análisis funcional}
En este capítulo se detalla el funcionamiento del sistema. Primero se identificará a los actores que intervienen en el sistema, indicando los objetivos de cada uno. A continuación se presentarán los escenarios empleando casos de uso. Por último se detallarán las interacciones entre los actores y los demás elementos del sistema en los principales escenarios mediante diagramas de secuencia.

%%%%%%%%%%%%%%%%%%%%%%%%%%%%%%%%%%%%%%%%%%%%%%%%%%%%%%%%%%%%%%%%%%%%%%%%%%%%%%%%%%%%%%%%%%%%%%%%%%%
%%%%%%%%%%%%%%%%%%%%%%%%%%%%%%%%%%%%%%%%%%%%%%%%%%%%%%%%%%%%%%%%%%%%%%%%%%%%%%%%%%%%%%%%%%%%%%%%%%%
%%%%%%%%%%%%%%%%%%%%%%%%%%%%%%%%%%%%%%%%%%%%%%%%%%%%%%%%%%%%%%%%%%%%%%%%%%%%%%%%%%%%%%%%%%%%%%%%%%%
%%%%%%%%%%%%%%%%%%%%%%%%%%%%%%%%%%%%%%%%%%%%%%%%%%%%%%%%%%%%%%%%%%%%%%%%%%%%%%%%%%%%%%%%%%%%%%%%%%%
%%%%%%%%%%%%%%%%%%%%%%%%%%%%%%%%%%%%%%%%%%%%%%%%%%%%%%%%%%%%%%%%%%%%%%%%%%%%%%%%%%%%%%%%%%%%%%%%%%%
\section{Identificación de actores}
Como ya se introdujo en \ref{objetivos-especificos-gestor-contenidos}, en \emph{PushNews} existen tres tipos de actores o tres perfiles. A continuación se profundizará en los objetivos de cada uno de ellos.

\subsection{Lector}
Cualquier usuario que, incluso sin estar previamente registrado y sin haberse identificado en el sistema, consulte una comunicación. Los principales objetivos de los lectores son:
\begin{enumerate}
    \item Visualizar las comunicaciones publicadas.
    \item Obtener los recursos adjuntos a las comunicaciones (imágenes, documentos\dots).
    \item Saber cuando una nueva comunicación es publicada.
\end{enumerate}

\subsection{Editor}
Generalmente, los editores pertenecerán a la organizaciones de los clientes y su misión es publicar y mantener las comunicaciones así como consultar las métricas. Sus objetivos más importantes son:
\begin{enumerate}
    \item Publicar comunicaciones
    \item Gestión o mantenimiento de las comunicaciones (modificación, eliminación\dots)
    \item Gestión de las categorías de comunicaciones de la aplicación.
    \item Gestión de otros datos de las aplicaciones como lugares de interés, teléfonos, etc.
    \item Ver las métricas de accesos a las comunicaciones
\end{enumerate}

\subsection{Administrador}
Los administradores se encargan del mantenimiento del sistema desde la organización de \emph{PushNews}. Entre sus objetivos principales encontramos:
\begin{enumerate}
    \item Mantenimiento general del servicio.
    \item Administrar las aplicaciones (altas, bajas, etc\dots).
    \item Administrar las características asociadas a las aplicaciones.
    \item Administración de los usuarios (editores) de las aplicaciones. 
    \item Realizar labores de editor de cualquier aplicación.
\end{enumerate}

%%%%%%%%%%%%%%%%%%%%%%%%%%%%%%%%%%%%%%%%%%%%%%%%%%%%%%%%%%%%%%%%%%%%%%%%%%%%%%%%%%%%%%%%%%%%%%%%%%%
%%%%%%%%%%%%%%%%%%%%%%%%%%%%%%%%%%%%%%%%%%%%%%%%%%%%%%%%%%%%%%%%%%%%%%%%%%%%%%%%%%%%%%%%%%%%%%%%%%%
%%%%%%%%%%%%%%%%%%%%%%%%%%%%%%%%%%%%%%%%%%%%%%%%%%%%%%%%%%%%%%%%%%%%%%%%%%%%%%%%%%%%%%%%%%%%%%%%%%%
%%%%%%%%%%%%%%%%%%%%%%%%%%%%%%%%%%%%%%%%%%%%%%%%%%%%%%%%%%%%%%%%%%%%%%%%%%%%%%%%%%%%%%%%%%%%%%%%%%%
%%%%%%%%%%%%%%%%%%%%%%%%%%%%%%%%%%%%%%%%%%%%%%%%%%%%%%%%%%%%%%%%%%%%%%%%%%%%%%%%%%%%%%%%%%%%%%%%%%%
\section{Análisis de casos de uso}
Larman define caso de uso como ``una colección de escenarios con éxito y fallo relacionados, que describe a los actores utilizando un sistema para satisfacer un objetivo.'' \cite{Larman2004}. En este apartado se presentarán algunos casos de uso del sistema que requieren una atención especial y se omitirán los que, por su simplicidad, quedan suficientemente descritos en el apartado \ref{requisitos-funcionales} de requisitos funcionales.

\subsection{Consulta de una comunicación}

\subsubsection*{Descripción}
Un lector consulta un comunicado. El sistema contabiliza la consulta para las estadísticas de acceso.

\subsubsection*{Actores}
\begin{itemize}
    \item Lector: quiere leer el contenido del comunicado.
    \item Sistema: tiene que registrar el acceso para que se refleje en las estadísticas.
\end{itemize}

\subsubsection*{Precondiciones}
\begin{itemize}
    \item El comunicado existe y está activo y publicado.
\end{itemize}

\subsubsection*{Descripción}

\begin{enumerate}
    \item El lector solicita la lectura de un comunicado.
    \item El sistema carga el comunicado.
    \item El sistema comprueba que el comunicado está activo y publicado.
    \item El sistema comprueba si el terminal del lector está dado de alta.
    \begin{enumerate}
        \item Si el terminal está dado de alta actualiza la fecha de último acceso.
        \item Si el terminal no está dado de alta, lo registra.
    \end{enumerate}
    \item El sistema registra el acceso del terminal al comunicado.
    \item El sistema muestra el contenido del comunicado al lector.
\end{enumerate}

\subsubsection*{Escenarios alternativos}
\begin{itemize}
    \item Si el comunicado no existe, no está activo o no está publicado, el lector verá un mensaje de error y este suceso quedará reflejado en el log.
    \item Si se produce un error al actualizar los datos del terminal o del acceso al comunicado, el lector verá el contenido del comunicado igualmente pero este suceso quedará reflejado en el log.
\end{itemize}

\subsubsection*{Postcondiciones}
\begin{itemize}
    \item El comunicado se muestra al lector.
    \item La visita queda contabilizada en el sistema.
\end{itemize}

%%%%%%%%%%%%%%%%%%%%%%%%%%%%%%%%%%%%%%%%%%%%%%%%%%%%%%%%%%%%%%%%%%%%%%%%%%%%%%%

\subsection{Publicación de una comunicación}

\subsubsection*{Descripción}
Un editor escribe un comunicado en la web y el sistema lo registra y envía las notificaciones correspondientes.

\subsubsection*{Actores}
\begin{itemize}
    \item Editor: quiere publicar un comunicado
    \item Sistema: tiene que registrar el comunicado y enviar las notificaciones oportunas.
\end{itemize}

\subsubsection*{Precondiciones}
\begin{itemize}
    \item El editor está registrado el sistema y asociado a la aplicación en la que se publica el comunicado.
    \item La aplicación está activa.
\end{itemize}

\subsubsection*{Descripción}

\begin{enumerate}
    \item El editor crea un comunicado inmediato (con fecha-hora de publicación igual o anterior a la fecha y hora actuales).
    \item El sistema registra el comunicado.
    \item El sistema comprueba la fecha de publicación.
    \begin{enumerate}
        \item Si la fecha es anterior o igual a la actual (publicación instantánea) se envía el mensaje push y se marca la comunicación con el flag de push enviada y la fecha de envío.
        \item Si la publicación no es inmediata
    \end{enumerate}
    \item El sistema envía la notificación push.
    \item El sistema actualiza el comunicado con la fecha de envío de la notificación push y la marca de push enviada.
\end{enumerate}

\subsubsection*{Escenarios alternativos}
\begin{itemize}
    \item Si el editor ha añadido archivo adjunto o imagen, estos se cargan, se almacenan en disco y se registran en la tabla de Documentos para posteriormente ser vinculados a la comunicación al ser creada.
    \item Si el editor especifica un recordatorio, el sistema enviará una segunda notificación push una vez llegada la fecha y hora del recordatorio.
\end{itemize}

\subsubsection*{Postcondiciones}
\begin{itemize}
    \item El comunicado queda registrado y aparece en la rejilla de comunicados en la web.
\end{itemize}