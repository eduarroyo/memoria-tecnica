\chapter{Diseño}

La solución constará de 
En este apartado se describirá, la arquitectura de la solución, cada una de las partes de las que consta y la interfaz de usuario.

\section{Patrón MVC}

Modelo-vista-controlador (MVC) es un patrón de arquitectura de software que separa los datos y principalmente lo que es la lógica de negocio de una aplicación de su representación y el módulo encargado de gestionar los eventos y las comunicaciones. Para ello MVC propone la construcción de tres componentes distintos que son el modelo, la vista y el controlador, es decir, por un lado define componentes para la representación de la información, y por otro lado para la interacción del usuario. Este patrón de arquitectura de software se basa en las ideas de reutilización de código y la separación de conceptos, características que buscan facilitar la tarea de desarrollo de aplicaciones y su posterior mantenimiento \cite{wiki-mvc}.

\begin{figure}[h]
    \centering
    \begin{tikzpicture}[node distance=1cm, auto]  
        \tikzset{
            mynode/.style={rectangle,rounded corners,draw=black, top color=white, bottom color=yellow!50,very thick, inner sep=1em, minimum size=3em, text centered},
            myarrow/.style={->, >=latex', shorten >=1pt, thick},
            mylabel/.style={text width=7em, text centered} 
        }
        \node[mynode] (controlador) {Controlador};
        \node[below=3cm of controlador] (dummy) {}; 
        \node[mynode, left=of dummy] (vista) {Vista};
        \node[mynode, right=of dummy] (modelo) {Modelo};  
        
        \draw[myarrow] (controlador.south) -- (modelo.north);	
        \draw[myarrow] (vista.north) -- (controlador.south);
        \draw[myarrow] (modelo.west) -- (vista.east);
    \end{tikzpicture}
    \medskip
    \caption{Esquema de la arquitectura Modelo-Vista-Controlador} 
\end{figure}

\section{Bibliotecas de servicios}

Soportan la lógica de negocio, el esquema de autenticación y autorización y el acceso a datos. Enfoque por tipo de entidad.

\subsection{Modelo de datos de persistencia}

ORM Entity Framework, enfoque Code First, Migrations

\subsection{Operaciones de negocio}

\section{Aplicación WEB}

\subsection{Filtros de acción}

\subsection{Controladores}

\subsection{Modelos}

\subsection{Vistas}


\section{La parte cliente}

kendo, solicitudes ajax

\subsection{Área pública}

\subsection{Área privada}


Autenticación con Windows Identity Foundation.

\section{Servicio WEB}

\subsection{WEBAPI}

\subsection{Seguridad}

\section{Aplicación móvil tipo}