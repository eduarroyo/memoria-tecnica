\chapter{Objetivos}

El objetivo general de este trabajo es la creación de una plataforma software a modo de gestor de contenidos que permita a diferentes organizaciones realizar la gestión básica y seguimiento del impacto de sus comunicaciones corporativas. Los contenidos de este gestor serán consumidos por aplicaciones móviles derivadas de un mismo prototipo, lo que permitirá que sean producidas en poco tiempo y a coste reducido sin dejar de ser totalmente personalizadas.

Para la consecución de dichos objetivos, se establecen los siguientes objetivos específicos:

\section{Plataforma de publicación y seguimiento de comunicados}\label{objetivos-especificos-gestor-contenidos}
Es necesario dotar al sistema de una herramienta que permita a los editores preparar, publicar, programar y  consultar el impacto de los comunicados. Para ello se desarrollará una aplicación web que cubrirá las siguientes necesidades según el tipo de usuario:
\begin{itemize}
    \item Lectores (acceso público)
    \begin{itemize}
        \item Consulta pública de comunicados de la organización.
        \item Consulta de otros datos de interés de la organización (teléfonos, direcciones, etc.).
    \end{itemize}
    \item Editores (acceso restringido por organización)
    \begin{itemize}
        \item Administración de comunicados.
        \item Consulta de métricas de impacto de comunicaciones.
        \item Administración de tipos de registro independientes como categorías de comunicaciones, teléfonos, direcciones, etc. de la organización.
    \end{itemize}
    \item Administradores (responsables del mantenimiento del servicio)
    \begin{itemize}
        \item Administración de organizaciones.
        \item Administración de funcionalidades personalizadas disponibles para cada organización.
        \item Administración de usuarios.
        \item Configuración del servicio.
    \end{itemize}
\end{itemize}

\section{Servicio WEB de consulta de comunicados}
Para brindar acceso a los comunicados a través de aplicaciones propias o de terceros, es necesario dotar a la plataforma de un servicio WEB con una API pública que ofrezca los métodos necesarios para realizar dicha labor.
\begin{itemize}
    \item Lectores (acceso público)
    \begin{itemize}
        \item Consulta pública de comunicados de la organización.
        \item Consulta de otros datos de interés de la organización (teléfonos, direcciones, etc.).
    \end{itemize} 
\end{itemize}

\section{Aplicación móvil tipo}
Se desarrollará un prototipo aplicación móvil que consumirá el servicio web de la plataforma mediante solicitudes HTTP. Este prototipo será precursor de las aplicaciones finales que se desarrollen en el futuro para clientes y al mismo tiempo permitirá demostrar las capacidades del sistema.