\chapter{Especificación de requisitos}
En este apartado se identificarán los requisitos del sistema. Los requisitos son una descripción de las necesidades o deseos del producto \cite{Larman2004} y pueden ser de dos tipos:
\begin{enumerate}
    \item \textbf{Funcionales} son los que tratan sobre las funciones que debe realizar la aplicación.
    \item \textbf{No funcionales} son los que definen los criterios de calidad del sistema, como pueden ser rendimiento, extensibilidad, usabilidad\dots
\end{enumerate}

\section {Actores}
\emph{PushNews} ofrece diferentes funciones dependiendo del perfil del usuario que esté accediendo. Por lo tanto, antes de presentar la lista de requisitos clasificados, es interesante identificar los diferentes perfiles de usuario.

\subsection{Lector}
Un lector puede ser cualquier usuario, incluso sin estar previamente registrado o identificado en el sistema. En las organizaciones que así lo requieran, los lectores no podrán ser anónimos, sino que deberán estar registrados como asociados. Los objetivos de un lector son encontrar y visualizar las comunicaciones publicadas y sus recursos adjuntos (fotografías, documentos, mapas, etc.).

\subsection{Editor}
Un editor se encarga de redactar y publicar comunicaciones, tiene acceso a las métricas de éstas y también puede administrar algunos aspectos de las organizaciones de las que forme parte. Debe ser un usuario registrado en el sistema e identificarse antes de tener acceso a las funciones propias de su perfil.

\subsection{Administrador}
El administrador es el encargado del mantenimiento del sistema. Gestiona y configura el servicio, administra las organizaciones y puede actuar como editor de cualquier organización.

\section{Descripción modular}
Conceptualmente, el sistema PushNews se puede descomponer en varios módulos que agrupan funcionalidades relacionadas entre si. A continuación se describen los módulos en que se divide el sistema.

\subsection*{Comunicaciones}
Las comunicaciones son el concepto fundamental del sistema. La comunicación es una pieza de contenido que una organización publica para informar a sus lectores. El módulo de comunicaciones engloba todas las funcionalidades relacionadas con este concepto como la creación, publicación, consulta, conteo de visitas, etc.

\subsection*{Usuarios}
En este módulo recae la responsabilidad de definir quién tiene acceso a la parte privada del sistema y qué funciones puede realizar cada uno dentro de éste. Algunas de sus funciones son la creación de usuarios, cambio de perfil de un usuario, cambio de clave de acceso, etc.

\subsection*{Organizaciones}
Cada organización representa un cliente, una aplicación móvil en el mercado. El módulo de organizaciones es el responsable de registrar los datos necesarios de cada una de ellas como las ApiKeys, las URLs de la tienda de aplicaciones, el nombre o el logotipo. Además, de la organización depende la mayoría de las operaciones del sistema, que se realizan dentro del ámbito de una organización.

\subsection*{Características de aplicación}
El módulo de características de aplicación es el encargado de definir el conjunto de funcionalidades opcionales disponibles para las organizaciones que lo soliciten.

\subsection*{Módulos de listado genérico}
Es un tipo de módulo para realizar las operaciones básicas de listado, creación, eliminación y modificación sobre un tipo de entidad cualquiera que no tenga procesos de negocio asociados. En esta definición entran módulos como el de lista de teléfonos de interés o el de lista de localizaciones geográficas.

\section{Requisitos funcionales}
En esta sección se expondrá la relación de requisitos funcionales del sistema clasificados módulo. Además, para cada requisito se indicará el perfil o los perfiles de usuario que participan en la operación: ``A'' para administrador, ``E'' para editor y ``L'' para lector.

\subsection{Requisitos funcionales de seguridad}
\begin{table}[hbt!]
  \centering
  \setlength\extrarowheight{14pt}
  \begin{tabularx}{\textwidth}{cXc}
    RF101\label{RF101} & Acceso de usuarios mediante correo electrónico y contraseña & A, E \\
    RF102\label{RF102} & Cambio de la propia contraseña. & A, E \\
  \end{tabularx}
  \label{cuadro:requistios-de-seguridad }
\end{table}
\FloatBarrier

\subsection{Requisitos funcionales de comunicaciones}
\begin{table}[hbt!]
  \centering
  \setlength\extrarowheight{14pt}
  \begin{tabularx}{\textwidth}{cXc}
    RF201\label{RF201} & Listado de las comunicaciones publicadas de la organización actual, mostrando únicamente los atributos públicos de las comunicaciones. & L \\
    RF202\label{RF202} & Listado de las comunicaciones publicadas de la organización actual, mostrando además los atributos no públicos como por ejemplo el estado de las notificaciones, el nº de visitas, etc. & A, E \\
    RF203\label{RF203} & Consulta pública de una comunicación publicada. Cada vez que se realice esta operación, se contabilizará la visita de cara a las métricas de impacto de la comunicación. & L \\
    RF204\label{RF203} & Previsualización de comunicación sin que esta operación se contabilice de cara a las métricas de impacto. & A, E \\
    RF204\label{RF204} & Publicación de comunicaciones en redes sociales. & L \\
    RF205\label{RF205} & Creación de comunicaciones dentro de una organización y publicación instantánea. & A, E \\
    RF206\label{RF206} & Creación de comunicaciones con publicación programada para un momento posterior. & A, E \\
    RF207\label{RF207} & Modificación del contenido de una publicación. & A, E \\
    RF208\label{RF208} & Eliminación de una publicación. & A, E \\
    RF209\label{RF209} & Desactivación/Activación de una comunicación. Una publicación desactivada no es visible públicamente. & A, E \\
    RF210\label{RF210} & Consulta de métricas de impacto de comunicaciones. & A, E
  \end{tabularx}
  \label{cuadro:requistios-de-comunicaciones }
\end{table}
\FloatBarrier

\subsection{Requisitos funcionales de usuarios}
\begin{table}[hbt!]
  \setlength\extrarowheight{14pt}
  \centering
  \begin{tabularx}{\textwidth}{cXc}
    RF301\label{RF301} & Listado de los usuarios registrados en el sistema. & A \\
    RF302\label{RF302} & Creación de usuarios. & A \\
    RF303\label{RF303} & Desactivación/Activación de un usuario. Un usuario desactivado no puede acceder al sistema. & A \\
    RF304\label{RF304} & Modificación de los datos de un usuario. Por ejemplo el nombre, el correo electrónico y también el perfil o las aplicaciones asociadas. & A \\
    RF305\label{RF305} & Eliminación de un usuario. & A \\
    RF307\label{RF307} & Cambio de clave de un usuario. & A \\
  \end{tabularx}
  \label{cuadro:requisitos-de-usuarios}
\end{table}
\FloatBarrier

\subsection{Requisitos funcionales de organizaciones}
\begin{table}[hbt!]
  \centering
  \setlength\extrarowheight{14pt}
  \begin{tabularx}{\textwidth}{cXc}
    RF401\label{RF401} & Listado de organizaciones. & A \\
    RF402\label{RF402} & Creación de organizaciones. & A \\
    RF403\label{RF403} & Activación/Desactivación de una organización. Una organización desactivada no es accesible para los usuarios. & A \\
    RF404\label{RF404} & Modificación de una organización. & A \\
  \end{tabularx}
  \label{cuadro:requisitos-de-organizaciones}
\end{table}
\FloatBarrier

\subsection{Requisitos funcionales de características debe aplicaciones}
\begin{table}[hbt!]
  \centering
  \setlength\extrarowheight{14pt}
  \begin{tabularx}{\textwidth}{cXc}
    RF401\label{RF401} & Listado de características & A, E \\
    RF403\label{RF403} & Activación/Desactivación. Una característica desactivada dejará de estar disponible en las aplicaciones que la tengan y no se podrá añadir a una organización. & A \\
  \end{tabularx}
  \label{cuadro:requisitos-de-aplicaciones}
\end{table}

\section{Requisitos no funcionales}