\chapter{Objetivos}
El objetivo general de este trabajo es la creación de una plataforma software que permita ofrecer en poco tiempo y a coste reducido aplicaciones móviles personalizadas asociadas a un backend que permita a diferentes organizaciones realizar la gestión básica y seguimiento del impacto de sus comunicaciones corporativas.

Como objetivo secundario, la plataforma deberá estar abierta a la incorporación de nuevas funcionalidades que cubran las necesidades concretas que un cliente necesite. Como ejemplo, se incluirán algunas características opcionales que se podrán activar para un determinado cliente.

Para la consecución de dichos objetivos, se establecen los siguientes objetivos específicos:

\section{Plataforma de publicación y seguimiento de comunicados}
Es necesario dotar al sistema de una herramienta que permita a los editores elaborar, publicar, programar (publicación diferida) y consultar las métricas de impacto de los comunicados. Para ello se desarrollará una aplicación web con acceso individualizado que cubrirá las siguientes necesidades según el tipo de usuario:
\begin{itemize}
    \item Lectores (acceso público)
    \begin{itemize}
        \item Consulta pública de comunicados de la organización.
        \item Consulta de otros datos de interés de la organización (teléfonos, direcciones, etc.).
    \end{itemize}
    \item Editores (acceso restringido por organización)
    \begin{itemize}
        \item Administración de comunicados.
        \item Consulta de métricas de impacto de comunicaciones.
        \item Administración de tipos de registro independientes como categorías de comunicaciones, teléfonos, direcciones, etc. de la organización.
    \end{itemize}
    \item Administradores (responsables del mantenimiento del servicio)
    \begin{itemize}
        \item Administración de organizaciones.
        \item Administración de funcionalidades personalizadas disponibles para cada organización.
        \item Administración de usuarios.
        \item Configuración del servicio.
    \end{itemize}
\end{itemize}

\section{Servicio WEB de consulta de comunicados}
Para brindar acceso a los comunicados a través de aplicaciones propias o de terceros, es necesario dotar a la plataforma de un servicio WEB con una API pública que ofrezca los métodos necesarios para realizar dicha labor.
\begin{itemize}
    \item Lectores (acceso público)
    \begin{itemize}
        \item Consulta pública de comunicados de la organización.
        \item Consulta de otros datos de interés de la organización (teléfonos, direcciones, etc.).
    \end{itemize}
\end{itemize}

\section{Aplicación móvil tipo}
Se desarrollará una aplicación móvil que permita consultar los comunicados de una organización desde su teléfono móvil, consumiendo el servicio web de la plataforma mediante solicitudes HTTP. Esta aplicación será un prototipo precursor de las aplicaciones para clientes y demostrará las capacidades del sistema.