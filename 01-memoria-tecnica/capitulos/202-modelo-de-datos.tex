\chapter{Modelo de datos}
En este capítulo se estudiará la estructura de datos del sistema mediante un \textit{Modelo Entidad-Interrelación} (E-R).
Primero se identificarán y detallarán los tipos de entidad, luego los tipos de interrelación y finalmente se sintetizará todo el modelo en un diagrama E-R.

\section {Análisis de los tipos de entidad}
En este apartado se definirán y describirán los tipos de entidad encontrados. 

\subsection{Aplicaciones}
\subsubsection*{Descripción}
Este tipo de entidad representa las aplicaciones móviles de pushnews. Por cada aplicación registrada habrá una aplicación móvil en la tienda de aplicaciones de Google, de Apple o en ambas. Cada aplicación cuenta con un subdominio propio para acceder a pushnews, unas credenciales para el servicio de mensajería push, su propio logotipo y una APIKEY para verificar la autenticidad de las solicitudes al servicio web.

\subsubsection*{Restricciones}
No habrá aplicaciones con el mismo nombre tampoco con el mismo subdominio.

\subsubsection*{Características}
\begin{description}[nosep,style=multiline,labelindent=0.8cm,leftmargin=4cm,font=\normalfont]
    \item[Nombre] Aplicaciones
    \item[Id. principal] AplicacionID
    \item[Id. alternativo] Subdominio
    \item[Atrib. heredados] LogotipoID (Documentos)
\end{description}

\subsubsection*{Atributos de la entidad}
En la tabla \ref{cuadro:atributos-tipo-entidad-aplicaciones} se describen todos los atributos de la entidad. Así mismo, en la tabla \ref{cuadro:ejemplo-aplicacion} se muestra un ejemplo de los valores que tendría un registro de aplicación.

\begin{table}[h!]
    \rowcolors{2}{gray!25}{white}
    \centering
    %\resizebox{\textwidth}{!}{%
    \begin{tabular}{|llcp{7.5cm}|}
        \hline
        \rowcolor[HTML]{9B9B9B}
        \multicolumn{1}{|l}{\cellcolor[HTML]{9B9B9B}{\color[HTML]{FFFFFF} Atributo}} & 
        \multicolumn{1}{c}{\cellcolor[HTML]{9B9B9B}{\color[HTML]{FFFFFF} Dominio}} &
        \multicolumn{1}{c}{\cellcolor[HTML]{9B9B9B}{\color[HTML]{FFFFFF} Obl.}} &
        \multicolumn{1}{c|}{\cellcolor[HTML]{9B9B9B}{\color[HTML]{FFFFFF} Descripción}} \\
        AplicacionID & $\mathbb N$ & \cmark & Identificador de aplicación \\
        Nombre & Alfanumérico & \cmark & Nombre de la aplicación \\
        Versión & Alfanumérico & \cmark & Versión de la aplicación \\
        Activo & Booleano & \cmark & Aplicación activa/inactiva \\
        SubDominio & Alfanumérico & \cmark & Subdominio de la aplicación \\
        CloudKey & Alfanumérico & \cmark & APIKEY del servicio de mensajería PUSH \\
        Usuario & Alfanumérico & \xmark & Usuario del servicio de mensajería PUSH \\
        Clave & Alfanumérico & \xmark & Contraseña del servicio de mensajería PUSH \\
        LogotipoID & $\mathbb N$ & \xmark & ID del documento con el logotipo \\
        ApiKey & Alfanumérico & \cmark & Clave de seguridad de la API.\\
        PlayStoreUrl & URL & \xmark & Url de la aplicación en la PlayStore \\
        AppStoreUrl & URL & \xmark & Url de la aplicación en la AppStore. \\
        \hline
    \end{tabular}%}
    \caption{Atributos del tipo de entidad Aplicaciones}
    \label{cuadro:atributos-tipo-entidad-aplicaciones}
\end{table} 

\begin{table}[h!]
    \rowcolors{2}{gray!25}{white}
    \centering
    %\resizebox{\textwidth}{!}{%
    \begin{tabular}{|ll|}
        \hline
        \rowcolor[HTML]{9B9B9B} 
        \multicolumn{1}{|c}{\cellcolor[HTML]{9B9B9B}{\color[HTML]{FFFFFF} Atributo}} & \multicolumn{1}{c|}{\cellcolor[HTML]{9B9B9B}{\color[HTML]{FFFFFF} Valor}} \\ \hline
        AplicacionID & 1 \\
        Nombre & ``Escuela Politécnica Superior de Córdoba'' \\
        Versión & ``1.0.0.0'' \\
        Activo & Verdadero \\
        SubDominio & ``epsc'' \\
        CloudKey & <texto encriptado> \\
        Usuario & epspush \\
        Clave & <texto encriptado> \\
        LogotipoID & 8 \\
        ApiKey & `AIzaSyCqhjgrPTPSOFyLyos5gfN47TJ0HnNA\_LA' \\
        PlayStoreUrl & `https://play.google.com/\dots?id=com.pushnews.epsc' \\
        AppStoreUrl & `https://apps.apple.com/\dots/pushnews-epsc' \\
        \hline
    \end{tabular}%}
    \caption{Ejemplo de registro de tipo Aplicación}
    \label{cuadro:ejemplo-aplicacion}
\end{table}

\subsection{Usuarios}

\subsubsection*{Descripción}
Los usuarios del sistema tienen un email único y una clave para acceder. Además, se almacena su nombre y una marca para indicar si el email ha sido confirmado.

Los usuarios tienen un rol de editor o administrador. Un editor se vincula a una o varias aplicaciones en las que puede administrar las comunicaciones, categorías, etc. Los administradores gestionan las aplicaciones existentes, usuarios, parámetros del sistema\dots y también pueden actuar como editores dentro de cualquier aplicación.

\subsubsection*{Restricciones}
No habrá usuarios con el mismo email.

\subsubsection*{Características}
\begin{description}[nosep,style=multiline,labelindent=0.8cm,leftmargin=4cm,font=\normalfont]
    \item[Nombre] Usuarios
    \item[Id. principal] UsuarioID
    \item[Id. alternativo] Email
    \item[Atrib. heredados] RolID (Roles.RolID)
\end{description}

\subsubsection*{Atributos de la entidad}
En la tabla \ref{cuadro:atributos-tipo-entidad-usuarios} se describen todos los atributos de la entidad. Así mismo, en la tabla \ref{cuadro:ejemplo-usuario} se muestra un ejemplo de los valores que tendría un registro de usuario.

\begin{table}[h!]
    \rowcolors{2}{gray!25}{white}
    \centering
    %\resizebox{\textwidth}{!}{%
    \begin{tabular}{|llcp{7.5cm}|}
        \hline
        \rowcolor[HTML]{9B9B9B}
        \multicolumn{1}{|l}{\cellcolor[HTML]{9B9B9B}{\color[HTML]{FFFFFF} Atributo}} & 
        \multicolumn{1}{c}{\cellcolor[HTML]{9B9B9B}{\color[HTML]{FFFFFF} Dominio}} &
        \multicolumn{1}{c}{\cellcolor[HTML]{9B9B9B}{\color[HTML]{FFFFFF} Obl.}} &
        \multicolumn{1}{c|}{\cellcolor[HTML]{9B9B9B}{\color[HTML]{FFFFFF} Descripción}} \\
        UsuarioID & $\mathbb N$ & \cmark & Identificador de usuario \\
        Email & Alfanumérico & \cmark & Email del usuario \\
        Nombre & Alfanumérico & \cmark & Nombre del usuario \\
        Apellidos & Alfanumérico & \cmark & Apellidos del usuario \\
        Clave & Alfanumérico & \cmark & Clave de acceso del usuario \\
        Activo & Booleano & \cmark & Aplicación activa/inactiva \\
        EmailConfirmado & Booleano & \cmark & El usuario ha confirmado el email \\
        Creado & Fecha & \cmark & Fecha de creación del registro \\
        Actualizado & Fecha & \xmark & Fecha de actualización del registro \\
        RolID & $\mathbb N$ & \cmark & ID del rol del usuario\\
        \hline
    \end{tabular}%}
    \caption{Atributos del tipo de entidad Usuarios}
    \label{cuadro:atributos-tipo-entidad-usuarios}
\end{table}


\begin{table}[h!]
    \rowcolors{2}{gray!25}{white}
    \centering
    %\resizebox{\textwidth}{!}{%
    \begin{tabular}{|ll|}
        \hline
        \rowcolor[HTML]{9B9B9B} 
        \multicolumn{1}{|c}{\cellcolor[HTML]{9B9B9B}{\color[HTML]{FFFFFF} Atributo}} & \multicolumn{1}{c|}{\cellcolor[HTML]{9B9B9B}{\color[HTML]{FFFFFF} Valor}} \\ \hline
        UsuarioID & 1 \\
        Email & ``joselopez@mailsrv.com'' \\
        Nombre & ``José'' \\
        Apellidos & ``López Pérez'' \\
        Clave & <texto encriptado> \\
        Activo & Verdadero \\
        EmailConfirmado & Falso \\
        Creado & 2020-11-07 11:03:00 \\
        Actualizado & 2020-11-08 16:52:00 \\
        RolID & 1 \\
        \hline
    \end{tabular}%}
    \caption{Ejemplo de registro de tipo Usuario}
    \label{cuadro:ejemplo-usuario}
\end{table}

\subsection{Comunicaciones}

\subsubsection*{Descripción}
Una comunicación es una publicación realizada por un editor en una aplicación. La comunicación contiene datos sobre su contenido como título, cuerpo, categoría y recursos adjuntos como enlaces, documentos o imágenes, fecha de publicación, etc. Además, tiene otros datos como su estado (publicada, borrada, activa...), estado de notificaciones push (notificada o no, recordatorio...). Las comunicaciones podrán ser inmediatas si se publican inmediatamente o programadas, cuando se programan para ser publicadas en una fecha y hora concretas.

\subsubsection*{Restricciones}
No habrá usuarios con el mismo email.

\subsubsection*{Características}
\begin{description}[nosep,style=multiline,labelindent=0.8cm,leftmargin=4cm,font=\normalfont]
    \item[Nombre] Comunicaciones
    \item[Id. principal] ComunicacionID
    \item[Id. alternativo] Ninguno
    \item[Atrib. heredados] UsuarioID (Usuarios), CategoriaID (Categorias), ImagenDocu\linebreak mentoID (Documentos), AdjuntoDocumentoID (Documentos)
\end{description}

\subsubsection*{Atributos de la entidad}
En la tabla \ref{cuadro:atributos-tipo-entidad-comunicaciones} se describen todos los atributos de la entidad. Así mismo, en la tabla \ref{cuadro:ejemplo-comunicacion} se muestra un ejemplo de los valores que tendría un registro de característica.

\begin{table}[ht]
    \rowcolors{2}{gray!25}{white}
    \centering
    %\resizebox{\textwidth}{!}{%
    \begin{tabular}{|llcp{7.2cm}|}
        \hline
        \rowcolor[HTML]{9B9B9B}
        \multicolumn{1}{|l}{\cellcolor[HTML]{9B9B9B}{\color[HTML]{FFFFFF} Atributo}} & 
        \multicolumn{1}{c}{\cellcolor[HTML]{9B9B9B}{\color[HTML]{FFFFFF} Dominio}} &
        \multicolumn{1}{c}{\cellcolor[HTML]{9B9B9B}{\color[HTML]{FFFFFF} Obl.}} &
        \multicolumn{1}{c|}{\cellcolor[HTML]{9B9B9B}{\color[HTML]{FFFFFF} Descripción}} \\
        ComunicacionID & $\mathbb N$ & \cmark & Identificador de la comunicación \\
        UsuarioID & $\mathbb N$ & \cmark & Identificador del usuario editor \\
        CategoriaID & $\mathbb N$ & \cmark & Identificador de la categoría de la comunicación \\
        FechaCreacion & Fecha & \cmark & Fecha de creación \\
        FechaPublicacion & Fecha & \cmark & Fecha de publicación \\
        Activo & Booleano & \cmark & Comunicación activa/inactiva \\
        Borrado & Booleano & \cmark & Marca de borrado de la comunicación \\
        FechaBorrado & Fecha & \xmark & Fecha de borrado de la comunicación \\
        TimeStamp & $\mathbb N$ & \xmark & Marca de tiempo de la publicación \\
        PushEnviada & Booleano & \cmark & Notificación push enviada/no enviada \\
        PushFecha & Fecha & \xmark & Fecha de notificación push \\
        Titulo & Alfanumérico & \cmark & Título de la comunicación \\
        Descripcion & Booleano & \cmark & Cuerpo de la comunicación \\
        UltimaEdicionIP & Alfanumérico & \xmark & IP desde la que se realizó la última edición \\
        Autor & Alfanumérico & \cmark & Nombre del autor \\
        Destacado & Booleano & \cmark & Publicación destacada/no destacada \\
        RecordatorioTitulo & Alfanumérico & \xmark & Texto del recordatorio \\
        RecordatorioFecha & Fecha & \xmark & Fecha del mensaje recordatorio \\
        PushRecordatorio & Booleano & \xmark & Mensaje push del recordatorio enviado/no enviado \\
        Instantanea & Booleano & \cmark & Publicación instantánea/diferida \\
        ImagenDocumentoID & $\mathbb N$ & \xmark & ID de la imagen adjunta \\
        ImagenTitulo & Alfanumérico & \xmark & Título de imagen adjunta \\
        AdjuntoDocumentoID & $\mathbb N$ & \xmark & ID del documento adjunto \\
        AdjuntoTitulo & Alfanumérico & \xmark & Título de documento adjunto \\
        EnlaceTitulo & Alfanumérico & \xmark & ID del rol del usuario \\
        Enlace & URL & \xmark & Enlace a recurso adjunto \\
        YoutubeTitulo & Alfanumérico & \xmark & Título del vídeo de Youtube adjunto \\
        Youtube & URL & \xmark & URL del video de Youtube adjunto \\
        GeoPosicionTitulo & Alfanumérico & \xmark & Título de la localización adjunta \\
        GeoPosicionLatitud & $\mathbb R$ & \xmark & Latitud de la localización adjunta \\
        GeoPosicionLongitud & $\mathbb R$ & \xmark & Longitud de la localización adjunta \\
        GeoPosicionDireccion & Alfanumérico & \xmark & Dirección de la localización adjunta \\
        GeoPosicionLocalidad & Alfanumérico & \xmark & Localidad de la localización adjunta \\
        GeoPosicionProvincia & Alfanumérico &\xmark & Provincia de la localización adjunta \\
        GeoPosicionPais & Alfanumérico & \xmark & País de la localización adjunta \\
        \hline
    \end{tabular}%}
    \caption{Atributos del tipo de entidad Comunicaciones}
    \label{cuadro:atributos-tipo-entidad-comunicaciones}
\end{table}

\begin{table}[ht]
    \rowcolors{2}{gray!25}{white}
    \centering
    %\resizebox{\textwidth}{!}{%
    \begin{tabular}{|ll|}
        \hline
        \rowcolor[HTML]{9B9B9B} 
        \multicolumn{1}{|c}{\cellcolor[HTML]{9B9B9B}{\color[HTML]{FFFFFF} Atributo}} & \multicolumn{1}{c|}{\cellcolor[HTML]{9B9B9B}{\color[HTML]{FFFFFF} Valor}} \\ \hline
        ComunicacionID & 19547 \\
        UsuarioID & 549 \\
        CategoriaID & 17 \\
        FechaCreacion & 2020-11-08 15:48:14 \\
        FechaPublicacion & 2020-11-10 10:00:00 \\
        Activo & Verdadero \\
        Borrado & Falso \\
        FechaBorrado & Nulo \\
        TimeStamp & 1604832148295 \\
        PushEnviada & Verdadero \\
        PushFecha & 2020-11-10 10:00:00 \\
        Titulo & ``RENOVACIÓN DE MATRÍCULA (GRADO)'' \\
        Descripcion & ``Se gestiona por Sigma según calendario\dots'' \\
        UltimaEdicionIP & 127.0.0.1 \\
        Autor & ``Eduardo Arroyo Ramírez'' \\
        Destacado & Falso \\
        RecordatorioTitulo & Nulo \\
        RecordatorioFecha & Nulo \\
        PushRecordatorio & Nulo \\
        Instantanea & Falso \\
        ImagenDocumentoID & 4981 \\
        ImagenTitulo & ``Calendario de renovaciones'' \\
        AdjuntoDocumentoID & Nulo \\
        AdjuntoTitulo & Nulo \\
        EnlaceTitulo & ``Solicitar renovación'' \\
        Enlace & ``http://uco.es/eps/es/renovacion-de-matricula-grado'' \\
        YoutubeTitulo & Nulo \\
        Youtube & Nulo \\
        GeoPosicionTitulo & ``Secretaría EPS'' \\
        GeoPosicionLatitud & 37.913484 \\
        GeoPosicionLongitud & -4.721551 \\
        GeoPosicionDireccion & Aulario Averroes, Rabanales \\
        GeoPosicionLocalidad & Córdoba \\
        GeoPosicionProvincia & Córdoba \\
        GeoPosicionPais & España \\
        \hline
    \end{tabular}%}
    \caption{Atributos del tipo de entidad Comunicaciones}
    \label{cuadro:ejemplo-comunicacion}
\end{table}

\subsection{Categorías}

\subsubsection*{Descripción}
Las categorías se definen para cada aplicación y agrupan las comunicaciones por temática. Tienen un nombre y un icono que las representa. Se ordenan según el criterio de los editores, para lo que cuentan con un campo orden. Pueden estar activas o inactivas. Si una categoría está inactiva, las comunicaciones de esta categoría no serán visibles ni se podrán crear nuevas comunicaciones en esta categoría.

\subsubsection*{Restricciones}
Ninguna

\subsubsection*{Características}
\begin{description}[nosep,style=multiline,labelindent=0.8cm,leftmargin=4cm,font=\normalfont]
    \item[Nombre] Usuarios
    \item[Id. principal] CategoriaID
    \item[Id. alternativo] Ninguno
    \item[Atrib. heredados] AplicacionID (Aplicaciones)
\end{description}

\subsubsection*{Atributos de la entidad}
En la tabla \ref{cuadro:atributos-tipo-entidad-categorias} se describen todos los atributos de la entidad. Así mismo, en la tabla \ref{cuadro:ejemplo-categoria} se muestra un ejemplo de los valores que tendría un registro de categoría.

\begin{table}[h!]
    \rowcolors{2}{gray!25}{white}
    \centering
    %\resizebox{\textwidth}{!}{%
    \begin{tabular}{|llcp{7.5cm}|}
        \hline
        \rowcolor[HTML]{9B9B9B}
        \multicolumn{1}{|l}{\cellcolor[HTML]{9B9B9B}{\color[HTML]{FFFFFF} Atributo}} & 
        \multicolumn{1}{c}{\cellcolor[HTML]{9B9B9B}{\color[HTML]{FFFFFF} Dominio}} &
        \multicolumn{1}{c}{\cellcolor[HTML]{9B9B9B}{\color[HTML]{FFFFFF} Obl.}} &
        \multicolumn{1}{c|}{\cellcolor[HTML]{9B9B9B}{\color[HTML]{FFFFFF} Descripción}} \\
        CategoriaID & $\mathbb N$ & \cmark & Identificador de la categoría \\
        AplicaciónID & $\mathbb N$ & \cmark & Identificador de la aplicación de la categoría \\
        Nombre & Alfanumérico & \cmark & Nombre de la categoría \\
        Icono & Alfanumérico & \cmark & Icono (fontawesome) de la categoría \\
        Orden & $\mathbb N$ & \cmark & Orden de la categoría \\
        Activo & Booleano & \cmark & Comunicación activa/inactiva \\
        \hline
    \end{tabular}%}
    \caption{Atributos del tipo de entidad Categorías}
    \label{cuadro:atributos-tipo-entidad-categorias}
\end{table}

\begin{table}[h!]
    \rowcolors{2}{gray!25}{white}
    \centering
    %\resizebox{\textwidth}{!}{%
    \begin{tabular}{|ll|}
        \hline
        \rowcolor[HTML]{9B9B9B} 
        \multicolumn{1}{|c}{\cellcolor[HTML]{9B9B9B}{\color[HTML]{FFFFFF} Atributo}} & \multicolumn{1}{c|}{\cellcolor[HTML]{9B9B9B}{\color[HTML]{FFFFFF} Valor}} \\ \hline
        CategoriaID & 17 \\
        AplicacionID & 3 \\
        Nombre & ``Secretaría'' \\
        Icono & ``fa-leanpub'' \\
        Orden & 3 \\
        Activo & Verdadero \\
        \hline
    \end{tabular}
    \caption{Ejemplo de registro de tipo Categorías}
    \label{cuadro:ejemplo-categoria}
\end{table}

\subsection{Características de aplicaciones}

\subsubsection*{Descripción}
Las características son funcionalidades que se pueden habilitar o deshabilitar para cada aplicación. Ejemplos de características serían la posibilidad de incrustar un vídeo de Youtube en las comunicaciones o el acceso al módulo de directorio comercial.

\subsubsection*{Restricciones}
No habrá dos características con el mismo nombre.

\subsubsection*{Características}
\begin{description}[nosep,style=multiline,labelindent=0.8cm,leftmargin=4cm,font=\normalfont]
    \item[Nombre] AplicacionesCaracteristicas
    \item[Id. principal] AplicacionCaracteristicaID
    \item[Id. alternativo] Ninguno
    \item[Atrib. heredados] Ninguno
\end{description}

\subsubsection*{Atributos de la entidad}
En la tabla \ref{cuadro:atributos-tipo-entidad-caracteristicas} se describen todos los atributos de la entidad. Así mismo, en la tabla \ref{cuadro:ejemplo-caracteristica} se muestra un ejemplo de los valores que tendría un registro de característica.

\begin{table}[h!]
    \rowcolors{2}{gray!25}{white}
    \centering
    %\resizebox{\textwidth}{!}{%
    \begin{tabular}{|llcp{5.9cm}|}
        \hline
        \rowcolor[HTML]{9B9B9B}
        \multicolumn{1}{|l}{\cellcolor[HTML]{9B9B9B}{\color[HTML]{FFFFFF} Atributo}} & 
        \multicolumn{1}{c}{\cellcolor[HTML]{9B9B9B}{\color[HTML]{FFFFFF} Dominio}} &
        \multicolumn{1}{c}{\cellcolor[HTML]{9B9B9B}{\color[HTML]{FFFFFF} Obl.}} &
        \multicolumn{1}{c|}{\cellcolor[HTML]{9B9B9B}{\color[HTML]{FFFFFF} Descripción}} \\
        AplicacionCaracteristicaID & $\mathbb N$ & \cmark & Identificador de la característica \\
        Nombre & Alfanumérico & \cmark & Nombre de la característica \\
        Activo & Booleano & \cmark & Característica activa/inactiva \\
        \hline
    \end{tabular}%}
    \caption{Atributos del tipo de entidad Características}
    \label{cuadro:atributos-tipo-entidad-caracteristicas}
\end{table}

\begin{table}[h!]
    \rowcolors{2}{gray!25}{white}
    \centering
    %\resizebox{\textwidth}{!}{%
    \begin{tabular}{|ll|}
        \hline
        \rowcolor[HTML]{9B9B9B} 
        \multicolumn{1}{|c}{\cellcolor[HTML]{9B9B9B}{\color[HTML]{FFFFFF} Atributo}} & \multicolumn{1}{c|}{\cellcolor[HTML]{9B9B9B}{\color[HTML]{FFFFFF} Valor}} \\ \hline
        AplicacionCaracteristicaID & 4 \\
        Nombre & ``DirectorioComercial'' \\
        Activo & Verdadero \\
        \hline
    \end{tabular}
    \caption{Ejemplo de registro de tipo Características}
    \label{cuadro:ejemplo-caracteristica}
\end{table}

\subsection{Terminales}

\subsection{Accesos}


\section {Análisis de los tipos de interrelación}

\subsection{Aplicación-Característica}

\subsection{Aplicación-Usuario}

\subsection{Aplicación-Categoría}

\subsection{Aplicación-Terminal}

\subsection{Categoría-Comunicación}

\subsection{Comunicaciones-Accesos-Terminales}

\section{Modelo Entidad-Interrelación}
