\chapter{Modelo de datos}
En este capítulo se estudiará la estructura de datos del sistema mediante un \textit{Modelo Entidad-Interrelación} (E-R).
Primero se identificarán y detallarán los tipos de entidad, luego los tipos de interrelación y finalmente se sintetizará todo el modelo en un diagrama E-R.

\section {Análisis de los tipos de entidad}
En este apartado se definirán y describirán los tipos de entidad encontrados. 

\subsection{Aplicaciones}
\subsubsection*{Descripción}
Almacena los datos sobre cada app asociada al sistema.
\subsubsection*{Restricciones}
No habrá aplicaciones con el mismo nombre ni el mismo subdominio.
\subsubsection*{Características}
\begin{description}[nosep,style=multiline,labelindent=0.8cm,leftmargin=5.5cm,font=\normalfont]
    \item[Nombre] Aplicaciones
    \item[Identificador principal] AplicacionID
    \item[Atributos heredados] LogotipoID (Documentos.DocumentoID)
    \item[Identificador alternativo] Subdominio
\end{description}

\subsubsection*{Atributos de la entidad}
En la tabla \ref{cuadro:atributos-tipo-entidad-aplicaciones} se describen todos los atributos de la entidad. Así mismo, en la tabla \ref{cuadro:ejemplo-aplicacion} se muestra un ejemplo de los valores que tendría un registro de aplicación.

\begin{table}[ht]
    \rowcolors{2}{gray!25}{white}
    \centering
    %\resizebox{\textwidth}{!}{%
    \begin{tabular}{|lccp{8.4cm}|}
        \hline
        \rowcolor[HTML]{9B9B9B}
        \multicolumn{1}{|l}{\cellcolor[HTML]{9B9B9B}{\color[HTML]{FFFFFF} Atributo}} & 
        \multicolumn{1}{c}{\cellcolor[HTML]{9B9B9B}{\color[HTML]{FFFFFF} Dominio}} &
        \multicolumn{1}{c}{\cellcolor[HTML]{9B9B9B}{\color[HTML]{FFFFFF} Obl.}} &
        \multicolumn{1}{c|}{\cellcolor[HTML]{9B9B9B}{\color[HTML]{FFFFFF} Descripción}} \\
        AplicacionID & N\textsuperscript{os} Naturales & \cmark & Identificador de aplicación \\
        Nombre & Alfanumérico & \cmark & Nombre de la aplicación \\
        Versión & Alfanumérico & \cmark & Versión de la aplicación \\
        Activo & Booleano & \cmark & Aplicación activa/inactiva \\
        SubDominio & Alfanumérico & \cmark & Subdominio de la aplicación \\
        CloudKey & Alfanumérico & \cmark & APIKEY del servicio de mensajería PUSH \\
        Usuario & Alfanumérico & \xmark & Usuario del servicio de mensajería PUSH \\
        Clave & Alfanumérico & \xmark & Contraseña del servicio de mensajería PUSH \\
        LogotipoID & n\textsuperscript{os} Naturales & \xmark & ID del documento con el logotipo \\
        ApiKey & Alfanumérico & \cmark & Clave de seguridad de la API.\\
        PlayStoreUrl & URL & \xmark & Url de la aplicación en la PlayStore \\
        AppStoreUrl & URL & \xmark & Url de la aplicación en la AppStore. \\
        \hline
    \end{tabular}%}
    \caption{Atributos del tipo de entidad Aplicaciones}
    \label{cuadro:atributos-tipo-entidad-aplicaciones}
\end{table} 

\begin{table}[ht]
    \rowcolors{2}{gray!25}{white}
    \centering
    %\resizebox{\textwidth}{!}{%
    \begin{tabular}{|ll|}
        \hline
        \rowcolor[HTML]{9B9B9B} 
        \multicolumn{1}{|c}{\cellcolor[HTML]{9B9B9B}{\color[HTML]{FFFFFF} Atributo}} & \multicolumn{1}{c|}{\cellcolor[HTML]{9B9B9B}{\color[HTML]{FFFFFF} Valor}} \\ \hline
        AplicacionID & 1 \\
        Nombre & ``Escuela Politécnica Superior de Córdoba'' \\
        Versión & ``1.0.0.0'' \\
        Activo & Verdadero \\
        SubDominio & ``epsc'' \\
        CloudKey & <texto encriptado> \\
        Usuario & epspush \\
        Clave & <texto encriptado> \\
        LogotipoID & 8 \\
        ApiKey & `AIzaSyCqhjgrPTPSOFyLyos5gfN47TJ0HnNA\_LA' \\
        PlayStoreUrl & `https://play.google.com/\dots?id=com.pushnews.epsc' \\
        AppStoreUrl & `https://apps.apple.com/\dots/pushnews-epsc' \\
        \hline
    \end{tabular}%}
    \caption{Ejemplo de registro de tipo Aplicación}
    \label{cuadro:ejemplo-aplicacion}
\end{table}

\subsection{Usuarios}

\subsubsection*{Descripción}
Almacena los datos de los usuarios del sistema.
\subsubsection*{Restricciones}
No habrá usuarios con el mismo email.
\subsubsection*{Características}
\begin{description}[nosep,style=multiline,labelindent=0.8cm,leftmargin=5.5cm,font=\normalfont]
    \item[Nombre] Usuarios
    \item[Identificador principal] UsuarioID
    \item[Atributos heredados] RolID (Roles.RolID)
    \item[Identificador alternativo] Email
\end{description}

\subsubsection*{Atributos de la entidad}
En la tabla \ref{cuadro:atributos-tipo-entidad-usuarios} se describen todos los atributos de la entidad. Así mismo, en la tabla \ref{cuadro:ejemplo-usuario} se muestra un ejemplo de los valores que tendría un registro de usuario.

\begin{table}[ht]
    \rowcolors{2}{gray!25}{white}
    \centering
    %\resizebox{\textwidth}{!}{%
    \begin{tabular}{|lccp{7.5cm}|}
        \hline
        \rowcolor[HTML]{9B9B9B}
        \multicolumn{1}{|l}{\cellcolor[HTML]{9B9B9B}{\color[HTML]{FFFFFF} Atributo}} & 
        \multicolumn{1}{c}{\cellcolor[HTML]{9B9B9B}{\color[HTML]{FFFFFF} Dominio}} &
        \multicolumn{1}{c}{\cellcolor[HTML]{9B9B9B}{\color[HTML]{FFFFFF} Obl.}} &
        \multicolumn{1}{c|}{\cellcolor[HTML]{9B9B9B}{\color[HTML]{FFFFFF} Descripción}} \\
        UsuarioID & N\textsuperscript{os} Naturales & \cmark & Identificador de usuario \\
        Email & Alfanumérico & \cmark & Email del usuario \\
        Nombre & Alfanumérico & \cmark & Nombre del usuario \\
        Apellidos & Alfanumérico & \cmark & Apellidos del usuario \\
        Clave & Alfanumérico & \cmark & Clave de acceso del usuario \\
        Activo & Booleano & \cmark & Aplicación activa/inactiva \\
        EmailConfirmado & Booleano & \cmark & El usuario ha confirmado el email \\
        Creado & Fecha & \cmark & Fecha de creación del registro \\
        Actualizado & Fecha & \xmark & Fecha de actualización del registro \\
        RolID & N\textsuperscript{os} Naturales & \cmark & ID del rol del usuario\\
        \hline
    \end{tabular}%}
    \caption{Atributos del tipo de entidad Usuarios}
    \label{cuadro:atributos-tipo-entidad-usuarios}
\end{table}


\begin{table}[ht]
    \rowcolors{2}{gray!25}{white}
    \centering
    %\resizebox{\textwidth}{!}{%
    \begin{tabular}{|ll|}
        \hline
        \rowcolor[HTML]{9B9B9B} 
        \multicolumn{1}{|c}{\cellcolor[HTML]{9B9B9B}{\color[HTML]{FFFFFF} Atributo}} & \multicolumn{1}{c|}{\cellcolor[HTML]{9B9B9B}{\color[HTML]{FFFFFF} Valor}} \\ \hline
        UsuarioID & 1 \\
        Email & ``joselopez@mailsrv.com'' \\
        Nombre & ``José'' \\
        Apellidos & ``López Pérez'' \\
        Clave & <texto encriptado> \\
        Activo & Verdadero \\
        EmailConfirmado & Falso \\
        Creado & 2020-11-07 11:03:00 \\
        Actualizado & 2020-11-08 16:52:00 \\
        RolID & 1 \\
        \hline
    \end{tabular}%}
    \caption{Ejemplo de registro de tipo Usuario}
    \label{cuadro:ejemplo-usuario}
\end{table}

\subsection{Comunicaciones}

\subsection{Categorías}

\subsection{Características}

\subsection{Terminales}

\subsection{Accesos}


\section {Análisis de los tipos de interrelación}

\subsection{Aplicación-Característica}

\subsection{Aplicación-Usuario}

\subsection{Aplicación-Categoría}

\subsection{Aplicación-Terminal}

\subsection{Categoría-Comunicación}

\subsection{Comunicaciones-Accesos-Terminales}

\section{Modelo Entidad-Interrelación}
