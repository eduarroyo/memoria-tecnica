\documentclass[a4paper,12pt,twoside,final]{scrbook}
\usepackage[utf8]{inputenc}
\usepackage[spanish,es-nodecimaldot]{babel} % International, rec. by RAE: http://www.tex-tipografia.com/marca_decimal.html

\usepackage{booktabs}
\usepackage[hidelinks]{hyperref}

\usepackage{tgpagella} % text only
\usepackage{mathpazo}  % math & text

\usepackage{tabularx}
\usepackage{placeins}

\usepackage{setspace}
\spacing{1.2}

% COLOR
\usepackage[usenames,dvipsnames,svgnames,table]{xcolor}
% Colores para estilo Proyecto Docente (tonos naranjas)
\definecolor{lightback}{HTML}{F4E0BF}
     \definecolor{back}{HTML}{F3C591}
\definecolor{lightline}{HTML}{FCAF5F}
     \definecolor{line}{HTML}{ED7900}
% Colores para portada
\definecolor{epsc:oscuro}{HTML}{280091}
 \definecolor{epsc:medio}{HTML}{4C5CC5}
 \definecolor{epsc:claro}{HTML}{3FCFD5}
 \definecolor{epsc:verde}{HTML}{00B299}

\usepackage{verbatim}
\usepackage{tikz} % used in cover to place images
\usetikzlibrary{babel,arrows,positioning,fit}
\usepackage[simplified]{pgf-umlcd}
\usepackage{pgf-umlsd}

\usepackage{datetime} % allow formal date format
% "Month, YEAR" date format, in spanish with the month uppercased not interfering other dates
\newcommand\Monthname[1][EMPTY]{
  \ifnum1=#1Enero\else
  \ifnum2=#1Febrero\else
  \ifnum3=#1Marzo\else
  \ifnum4=#1Abril\else
  \ifnum5=#1Mayo\else
  \ifnum6=#1Junio\else
  \ifnum7=#1Julio\else
  \ifnum8=#1Agosto\else
  \ifnum9=#1Septiembre\else
  \ifnum10=#1Octubre\else
  \ifnum11=#1Noviembre\else
  \ifnum12=#1Diciembre\else
  \fi\fi\fi\fi\fi\fi\fi\fi\fi\fi\fi\fi
}
\newdateformat{monthyeardate}{%
  \Monthname[\THEMONTH], \THEYEAR}


% FONTs
\usepackage[T1]{fontenc}
\usepackage{textcomp}           % Needed for new symbols like € ?

\usepackage[scaled]{berasans} % Font for the cover similar to Vera 33
%\renewcommand*\familydefault{\sfdefault}  %% To use as the base font of the document is to be sans serif


% PAGE STYLE
\usepackage[twoside,bindingoffset=0mm,headheight=16pt,margin=25mm]{geometry} %,verbose,showframe
%\usepackage{fancyhdr} % Encabezados
%\pagestyle{fancy}
% XXX Evitar el binding en la portada pero no en el resto del documento

% TABLES AND FIGURES
\usepackage{graphicx}
\graphicspath{ {img/} }

% SUBRALLADOS
\usepackage[normalem]{ulem}

\usepackage{longtable}



\begin{document}
%\frontmatter

%------------- Cover --------------
\thispagestyle{empty}

% Backgroud images
\begin{tikzpicture}[remember picture, overlay]
  % Top
  \node [anchor=north east, inner sep=0pt]  at (current page.north east)
     {\includegraphics[height=6cm]{topRightCorner.pdf}};
  % Bottom
  \node [anchor=south west, inner sep=0pt]  at (current page.south west)
     {\includegraphics[height=6cm]{bottomLeftCorner.pdf}};
  \node (uco) [anchor=south east, inner sep=0pt, xshift=-10mm, yshift=10mm]  at (current page.south east)
        {\includegraphics[height=2cm]{uco_debajo.pdf}};
  \node [anchor=south east, inner sep=0pt, xshift=-10mm]  at (uco.south west)
% Uncomment the chosen logo and comment the others:
        {\includegraphics[height=2cm]{emblema-ing-informatica.pdf}};
%        {\includegraphics[height=2cm]{emblema-ing-industrial.pdf}};
%        {\includegraphics[height=2cm]{emblema-ing-tec-industrial.pdf}};
\end{tikzpicture}


\begin{center}
\fontfamily{\sfdefault}\selectfont
\vspace*{2cm}

\vfill
\vfill
\includegraphics[width=12.5cm]{LogotipoEPSC.pdf}
\vfill
\vfill

\large\textbf{\color{epsc:medio}
  TRABAJO FIN DE GRADO EN INGENIERÍA INFORMÁTICA
}
\vfill

\Huge\textbf{\color{epsc:oscuro}
  Plataforma web de gestión de comunicados
}
\vfill
\vfill

\Large\textbf{\color{epsc:verde}
  MEMORIA TÉCNICA
}
\vfill
\vfill

\large{\color{epsc:oscuro}Autor}\\
\textbf{\color{epsc:medio}{ Eduardo Arroyo Ramírez }}
\vfill

\large{\color{epsc:oscuro} Director }\\
\textbf{\color{epsc:medio} Domingo Ortiz Boyer}
\vfill



\textbf{\color{epsc:verde} \monthyeardate\today}
\vfill
\vfill
\vspace{2.7cm}
\end{center}

%-------------------------------------------------------------------------------
\clearpage

\thispagestyle{empty}
\pagecolor{white}

\cleardoublepage
%-------------------------------------------------------------------------------
\frontmatter
\tableofcontents
\listoffigures
\listoftables

\cleardoublepage
\mainmatter
\part{Presentación}
\chapter{Introducción}
Contexto sobre el origen de este proyecto: empresa pequeña queriendo entrar en el mercado de aplicaciones móviles de bajo coste y escalables.
\chapter{Definición del problema}

\section{La comunicación corporativa}
La \emph{Comunicación Corporativa} se define como <<la totalidad de los recursos de comunicación de los que dispone una organización para llegar efectivamente a sus públicos>> \cite{riel_2001}. Dentro de este concepto, se distingue entre la \emph{Comunicación Corporativa Interna} y la \emph{Comunicación Corporativa Externa} siendo la primera <<todo lo relativo a la conexión requerida entre los miembros de una determinada estructura para acometer una metas comunes>> y la segunda, <<la vinculación de la organización con el entorno en el que desarrolla sus actividades, con el fin de alcanzar un determinado nivel de rentabilidad económica y social>> \cite{castro_2007}. Desde el ámbito empresarial y administrativo se da cada vez mayor importancia a la comunicación ya que ésta juega un papel fundamental en la consecución de sus objetivos.

Este tipo de comunicación ha encontrado en Internet uno de sus canales principales y tanto empresas como instituciones ejercen hoy en día su presencia online a través de webs, blogs, redes sociales o incluso sus propias aplicaciones para móviles, que han proliferado enormemente durante los últimos años \cite{playstore} \cite{appstore}. Todas estas herramientas permiten no sólo la publicación de comunicados sino también la planificación temporal, seguimiento y medida del impacto.

\section{Aplicaciones para móviles}
Aunque las redes sociales son uno de los medios preferidos por las empresas \cite{linkedin} y administraciones \cite{grande2015casos} como canal de comunicación debido a la gran acogida que tienen entre la población, las aplicaciones móviles son una alternativa muy a tener en cuenta ya que cubren la frecuente necesidad de incluir otros servicios específicos de la organización como por ejemplo:
\begin{itemize}
    \item Sistema de reservas (mesas en un restaurante, salas en un gimnasio\dots).
    \item Encuestas (toma de decisiones en una comunidad de vecinos, sondeo de la clientela de un bar para organizar un concierto en directo\dots)
    \item Sistema de geolocalización en tiempo real de un desfile (¿por dónde está pasando ahora mismo la cabalgata de reyes?).
\end{itemize}
Además, las aplicaciones salvan algunos de los inconvenientes de las redes sociales como la cesión de datos, la alta exposición a la crítica o la obligación de adaptarse al diseño y las reglas de negocio impuestas por un tercero.
También es necesario mencionar el enorme desarrollo que ha tenido el acceso a internet desde dispositivos móviles en los últimos años, como se puede observar en la figura \ref{fig:acceso-internet-tipo-dispositivo} \cite{dispositivos_internet_2019}.

\begin{figure}[ht]
    \centering
    % https://www.ontsi.red.es/es/indicadores/Hogares-y-ciudadanos/Internet/Dispositivos-de-acceso-Internet
    \includegraphics[scale=.5]{AccesoAInternetPorTipoDispositivo2019.png}
    \caption{Personas que han usado Internet en los últimos tres meses por tipo de dispositivo utilizado }
    \label{fig:acceso-internet-tipo-dispositivo}
\end{figure}
\chapter{Objetivos}

El objetivo general de este trabajo es la creación de una plataforma software que permita ofrecer en poco tiempo y a coste reducido aplicaciones móviles personalizadas asociadas a un gestor de contenidos que permita a diferentes organizaciones realizar la gestión básica y seguimiento del impacto de sus comunicaciones corporativas.

Para la consecución de dichos objetivos, se establecen los siguientes objetivos específicos:

\section{Plataforma de publicación y seguimiento de comunicados}
Es necesario dotar al sistema de una herramienta que permita a los editores preparar, publicar, programar y  consultar el impacto de los comunicados. Para ello se desarrollará una aplicación web que cubrirá las siguientes necesidades según el tipo de usuario:
\begin{itemize}
    \item Lectores (acceso público)
    \begin{itemize}
        \item Consulta pública de comunicados de la organización.
        \item Consulta de otros datos de interés de la organización (teléfonos, direcciones, etc.).
    \end{itemize}
    \item Editores (acceso restringido por organización)
    \begin{itemize}
        \item Administración de comunicados.
        \item Consulta de métricas de impacto de comunicaciones.
        \item Administración de tipos de registro independientes como categorías de comunicaciones, teléfonos, direcciones, etc. de la organización.
    \end{itemize}
    \item Administradores (responsables del mantenimiento del servicio)
    \begin{itemize}
        \item Administración de organizaciones.
        \item Administración de funcionalidades personalizadas disponibles para cada organización.
        \item Administración de usuarios.
        \item Configuración del servicio.
    \end{itemize}
\end{itemize}

\section{Servicio WEB de consulta de comunicados}
Para brindar acceso a los comunicados a través de aplicaciones propias o de terceros, es necesario dotar a la plataforma de un servicio WEB con una API pública que ofrezca los métodos necesarios para realizar dicha labor.
\begin{itemize}
    \item Lectores (acceso público)
    \begin{itemize}
        \item Consulta pública de comunicados de la organización.
        \item Consulta de otros datos de interés de la organización (teléfonos, direcciones, etc.).
    \end{itemize} 
\end{itemize}

\section{Aplicación móvil tipo}
Se desarrollará un prototipo aplicación móvil que consumirá el servicio web de la plataforma mediante solicitudes HTTP. Este prototipo será precursor de las aplicaciones finales que se desarrollen en el futuro para clientes y al mismo tiempo permitirá demostrar las capacidades del sistema.
\chapter{Antecedentes}

\section{Introducción}
En este apartado se analizan brevemente herramientas existentes que, o bien son similares o bien tienen objetivos parecidos a \emph{PushNews}. Se han seleccionado herramientas de comunicación corporativa como boletines informativos o webs corporativas y también de marketing digital como las que se utilizan para gestionar redes sociales.

\section{Newsletter o boletín informativo}
Una newsletter es una publicación digital periódica que se distribuye a través de correo electrónico. Generalmente recopila los artículos o comunicados publicados desde el último envío. Los receptores de una newsletter son personas que han mostrado previamente interés en el tema y se han suscrito a la publicación para recibirla en su bandeja de correo electrónico. La newsletter sigue siendo a día de hoy una herramienta muy importante en la comunicación corporativa por varias razones según Wikipedia \cite{wiki_boletininformativo}:
\subsection*{Ventajas}
\begin{enumerate}
    \item Es un canal que, bien usado, permite construir una relación de confianza con el suscriptor que puede ser la base de futuras ventas.
    \item La calidad como lectores/clientes de las personas suscritas suele ser superior a los lectores esporádicos que encuentran llegan al sitio web, por ejemplo, a través de los buscadores.
    \item Porque es un activo que el autor de la lista tiene bajo su total control, a diferencia, por ejemplo, de los seguidores en redes sociales como Facebook o Twitter.
\end{enumerate}

\subsection*{Desventajas}
\begin{enumerate}
    \item Puede ser difícil conseguir que el que el usuario dé su consentimiento para recibir la newsletter.
    \item A menudo se considera spam.
    \item Es fácil que el usuario no preste atención a la newsletter entre todo el contenido de su bandeja de entrada.
\end{enumerate}

\section{Web corporativa}
Una web corporativa es un tipo de sitio web orientado a dar a conocer información sobre una empresa o institución. Aunque ya no supone un elemento diferenciador debido a que la mayoría de las organizaciones tienen una, resulta fundamental ya que proporciona la organización presencia en internet sin depender de terceros. Según The Company Warehouse \cite{web_corporativa}, las principales ventajas y desventajas de las webs corporativas son:
\subsection*{Ventajas}
\begin{enumerate}
    \item Potencialmente, llegan a una audiencia muy amplia.
    \item La información sobre la organización está siempre disponible y es fácilmente accesible.
    \item Grandes posibilidades de publicidad y marketing viral.
\end{enumerate}
\subsection*{Desventajas}
\begin{enumerate}
    \item Necesidad de mantener la información actualizada para no perder la confianza del público.
    \item Llegar a al público adecuado. Debido a la gran competencia existente, es muy difícil aparecer arriba en las búsquedas.
    \item Necesidad de mantener la web libre de errores y caídas de servicio.
\end{enumerate}

\section{Redes sociales}
Debido a su gran impacto social, las redes sociales se han convertido en un lugar obligado para las organizaciones donde ejercer su presencia digital gracias a sus posibilidades de segmentación, que permite llegar al público objetivo fácilmente. Según \cite{nadaraja2013socialmediamarketing}, estas son las principales ventajas y desventajas de las redes sociales:

\subsection*{Ventajas}
\begin{enumerate}
    \item Las barreras económicas son prácticamente inexistentes ya que la mayoría de redes sociales son de acceso gratuito. 
    \item La audiencia que desea recibir la información sigue las redes sociales de la organización y la comparten, ayudando a difundirla sin costes.
    \item La audiencia en las redes sociales está fuertemente segmentada, lo que permite llegar fácilmente a las personas realmente interesadas.
\end{enumerate}

\subsection*{Desventajas}
\begin{enumerate}
    \item Requieren mucho trabajo: debido a la naturaleza bidireccional de las interacciones en redes, atender a todas en cada una de las redes sociales supone una gan inversión de tiempo.
    \item Confianza, privacidad: al utilizar una plataforma de terceros, es obligatorio aceptar y someterse a las políticas de seguridad y privacidad de dicha plataforma.
    \item Respuesta negativa: dado que cualquier usuario de la red social puede generar contenido asociado al de una organización, esta se expone a recibir una respuesta negativa de la audiencia que tiene un gran impacto sobre su credibilidad.
\end{enumerate}

\subsection{Herramientas de gestión de redes sociales}
Existen soluciones que ayudan a gestionar las diferentes redes sociales con herramientas que facilitan las labores comunes del trabajo en redes sociales \cite{herramientas-social-media-1} \cite{herramientas-social-media-2} paliando algunas de las desventajas citadas anteriormente. Ejemplos de estas herramientas son Hootsuite, Keyhole, Narrow, Buffer. 

Las funcionalidades principales que aporta este tipo de herramientas son:
\begin{itemize}
    \item Publicar comunicados en varias redes a la vez.
    \item Planificar la publicación en el tiempo.
    \item Facilitar la gestión de las interacciones.
    \item Medir el impacto de los comunicados publicados.
\end{itemize}
\include{capitulos/104-limitaciones}
\include{capitulos/105-recursos}
\part{Análisis del sistema}
\chapter{Especificación de requisitos}
En este apartado se identificarán los requisitos del sistema. Los requisitos son una descripción de las necesidades o deseos del producto \cite{Larman2004}. Los requisitos pueden ser de dos tipos:
\begin{enumerate}
    \item \textbf{Funcionales} son los que tratan sobre las funciones que debe realizar la aplicación.
    \item \textbf{No funcionales} son los que definen los criterios de calidad del sistema, como pueden ser rendimiento, extensibilidad, usabilidad\dots
\end{enumerate}

\section {Actores}
\emph{PushNews} ofrece diferentes funciones dependiendo del tipo de usuario (perfil) que esté accediendo. Por lo tanto, antes de presentar la lista de requisitos clasificados, es interesante identificar los diferentes perfiles de usuario.

\subsection{Lector}
Un lector puede ser cualquier usuario, incluso sin estar previamente registrado o identificado en el sistema. En las organizaciones que así lo requieran, los lectores no podrán ser anónimos, sino que deberán estar registrados como asociados. Los objetivos de un lector son encontrar y visualizar las comunicaciones publicadas y sus recursos adjuntos (fotografías, documentos, mapas, etc.).

\subsection{Editor}
Un editor se encarga de redactar y publicar comunicaciones, tiene acceso a las métricas de éstas y también puede administrar algunos aspectos de las organizaciones de las que forme parte. Debe ser un usuario registrado en el sistema e identificarse antes de tener acceso a las funciones propias de su perfil.

\subsection{Administrador}
El administrador es el encargado del mantenimiento del sistema. Gestiona y configura el servicio, administra las organizaciones y puede actuar como editor de cualquier organización.

\section{Descripción modular}
En esta sección se realizará una descomposición del sistema en módulos conceptuales cada uno de los cuales se describirá brevemente.

\subsection*{Comunicaciones}
Las comunicaciones son el concepto fundamental del sistema. La comunicación es una pieza de contenido que una organización publica para informar a sus lectores. El módulo de comunicaciones engloba todas las funcionalidades relacionadas con este concepto como la creación, publicación, consulta, conteo de visitas, etc.

\subsection*{Usuarios}
En este módulo recae la responsabilidad de definir quién tiene acceso a la parte privada del sistema y qué funciones puede realizar cada uno dentro de éste. Algunas de sus funciones son la creación de usuarios, cambio de perfil de un usuario, cambio de clave de acceso, etc.

\subsection*{Organizaciones}
Cada organización representa un cliente, una aplicación móvil en el mercado. El módulo de organizaciones es el responsable de registrar los datos necesarios de cada una de ellas como las ApiKeys, las URLs de la tienda de aplicaciones, el nombre o el logotipo. Además, de la organización depende la mayoría de las operaciones del sistema, que se realizan dentro del ámbito de una organización.

\subsection*{Características de aplicación}
El módulo de características de aplicación es el encargado de definir el conjunto de funcionalidades opcionales disponibles para las organizaciones que lo soliciten.

\subsection*{Módulos de listado genérico}
Es un tipo de módulo para realizar las operaciones básicas de listado, creación, eliminación y modificación sobre un tipo de entidad cualquiera que no tenga procesos de negocio asociados. En esta definición entran módulos como el de lista de teléfonos de interés o el de lista de localizaciones geográficas.

\section{Requisitos funcionales}

\subsection{Funciones de seguridad}
\begin{table}[ht]
    \centering
    \begin{tabularx}{\textwidth}{|cX|}
    \rowcolor[HTML]{9B9B9B} 
    {\color[HTML]{FFFFFF} Ref \#} &
      \multicolumn{1}{l}{\cellcolor[HTML]{9B9B9B}{\color[HTML]{FFFFFF} Función}} \\ \hline
    R101\label{R101} & El sistema permitirá identificarse de forma segura mediante un correo electrónico y una contraseña \\
    R102\label{R102} & El sistema permitirá a un usuario identificado cambiar su contraseña.  \\
    R103\label{R}    & El sistema limitará el acceso de los usuarios a las funciones de su perfil (lector, editor, administrador) \\
    R104\label{R}    &  \\
    R105\label{R}    &  \\ \hline
    \end{tabularx}
    \caption{Funciones de seguridad}
    \label{cuadro:funciones-de-seguridad }
\end{table}


\subsection{Comunicaciones}
\begin{itemize}
  \item Listado público
  \item Consulta privada
  \item Consulta pública (contabilizar para estadísticas)
  \item Compartir en redes sociales
  \item Listado privado
  \item Creación
  \item Publicación diferida
  \item Modificación
  \item Eliminación
  \item Activación/Desactivación
\end{itemize}

\subsection{Usuarios}
\begin{itemize}
  \item Listado de usuarios
  \item Creación de usuario
  \item Activación/Desactivación
  \item Modificar un usuario
  \item Eliminar un usuario
  \item Cambiar contraseña propia
  \item Cambiar clave de un usuario
\end{itemize}

\subsection{Organizaciones}
\begin{itemize}
  \item Listado de organizaciones
  \item Creación de organización
  \item Activación/Desactivación
  \item Modificar una organización
  \item Cambiar características de aplicación
\end{itemize}

\subsection{Características de aplicaciones}
\begin{itemize}
  \item Listado de características
  \item Activación/Desactivación
\end{itemize}


\subsection{Funciones para lectores}
\begin{table}[ht]
    \centering
    \begin{tabularx}{\textwidth}{|cX|}
    \rowcolor[HTML]{9B9B9B} 
    {\color[HTML]{FFFFFF} Ref \#} &
      \multicolumn{1}{l}{\cellcolor[HTML]{9B9B9B}{\color[HTML]{FFFFFF} Función}} \\ \hline
    R201\label{R201} & Visualización de la lista de comunicaciones \\
    R202\label{R202} & Visualización de detalle de una comunicación \\
    R203\label{R203} & Descarga de adjuntos de una comunicación \\ 
    R204\label{R204} & Consulta de otros datos relacionados con la organización (teléfonos, localizaciones\dots) \\
    \hline
    \end{tabularx}
    \caption{Funciones para lectores}
    \label{cuadro:funciones-lectores }
\end{table}


\subsection{Funciones para editores}

\begin{table}[ht]
    \centering
    \begin{tabularx}{\textwidth}{|cX|}
    \rowcolor[HTML]{9B9B9B} 
    {\color[HTML]{FFFFFF} Ref \#} &
      \multicolumn{1}{l}{\cellcolor[HTML]{9B9B9B}{\color[HTML]{FFFFFF} Función}} \\ \hline
    R301\label{R301} & Administración de comunicaciones \\
    R302\label{R302} & Visualización de datos extendidos de comunicaciones (estado, fecha de creación, fecha de publicación\dots) \\
    R303\label{R303} & Administración de categorías de comunicaciones \\
    R304\label{R304} & Administración de otros datos de la aplicación (teléfonos, localizaciones\dots) \\ 
    R305\label{R305} & Consulta de métricas de los comunicaciones \\ 
    \hline
    \end{tabularx}
    \caption{Funciones para editores}
    \label{cuadro:funciones-editores }
\end{table}

\subsection{Funciones para administradores}

\begin{table}[ht]
    \centering
    \begin{tabularx}{\textwidth}{|cX|}
    \rowcolor[HTML]{9B9B9B} 
    {\color[HTML]{FFFFFF} Ref \#} &
      \multicolumn{1}{l}{\cellcolor[HTML]{9B9B9B}{\color[HTML]{FFFFFF} Función}} \\ \hline
    R401\label{R401} & Administración de aplicaciones \\
    R402\label{R402} & Administración de características de aplicaciones \\
    R403\label{R403} & Administración de configuración del servicio \\
    R404\label{R404} & Administración de usuarios \\ 
    R405\label{R405} & Consulta de métricas de las comunicaciones \\ 
    \hline
    \end{tabularx}
    \caption{Funciones para administradores}
    \label{cuadro:funciones-administradores}
\end{table}

\section{Requisitos no funcionales}
\chapter{Modelo de datos}
En este capítulo se estudiará la estructura de datos del sistema mediante un \textit{Modelo Entidad-Interrelación} (E-R).
Primero se identificarán y detallarán los tipos de entidad, luego los tipos de interrelación y finalmente se sintetizará todo el modelo en un diagrama E-R.

\section {Análisis de los tipos de entidad}
En este apartado se definirán y describirán los tipos de entidad encontrados. 

\subsection{Aplicaciones}
\subsubsection*{Descripción}
Este tipo de entidad representa las aplicaciones móviles de pushnews. Por cada aplicación registrada habrá una aplicación móvil en la tienda de aplicaciones de Google, de Apple o en ambas. Cada aplicación cuenta con un subdominio propio para acceder a pushnews, unas credenciales para el servicio de mensajería push, su propio logotipo y una APIKEY para verificar la autenticidad de las solicitudes al servicio web.

\subsubsection*{Restricciones}
No habrá aplicaciones con el mismo nombre tampoco con el mismo subdominio.

\subsubsection*{Características}
\begin{description}[nosep,style=multiline,labelindent=0.8cm,leftmargin=4cm,font=\normalfont]
    \item[Nombre] Aplicaciones
    \item[Id. principal] AplicacionID
    \item[Id. alternativo] Subdominio
    \item[Atrib. heredados] LogotipoID (Documentos)
\end{description}

\subsubsection*{Atributos de la entidad}
En la tabla \ref{cuadro:atributos-tipo-entidad-aplicaciones} se describen todos los atributos de la entidad. Así mismo, en la tabla \ref{cuadro:ejemplo-aplicacion} se muestra un ejemplo de los valores que tendría un registro de aplicación.

\begin{table}[h!]
    \rowcolors{2}{gray!25}{white}
    \centering
    %\resizebox{\textwidth}{!}{%
    \begin{tabular}{|llcp{7.5cm}|}
        \hline
        \rowcolor[HTML]{9B9B9B}
        \multicolumn{1}{|l}{\cellcolor[HTML]{9B9B9B}{\color[HTML]{FFFFFF} Atributo}} & 
        \multicolumn{1}{c}{\cellcolor[HTML]{9B9B9B}{\color[HTML]{FFFFFF} Dominio}} &
        \multicolumn{1}{c}{\cellcolor[HTML]{9B9B9B}{\color[HTML]{FFFFFF} Obl.}} &
        \multicolumn{1}{c|}{\cellcolor[HTML]{9B9B9B}{\color[HTML]{FFFFFF} Descripción}} \\
        AplicacionID & $\mathbb N$ & \cmark & Identificador de aplicación \\
        Nombre & Alfanumérico & \cmark & Nombre de la aplicación \\
        Versión & Alfanumérico & \cmark & Versión de la aplicación \\
        Activo & Booleano & \cmark & Aplicación activa/inactiva \\
        SubDominio & Alfanumérico & \cmark & Subdominio de la aplicación \\
        CloudKey & Alfanumérico & \cmark & APIKEY del servicio de mensajería PUSH \\
        Usuario & Alfanumérico & \xmark & Usuario del servicio de mensajería PUSH \\
        Clave & Alfanumérico & \xmark & Contraseña del servicio de mensajería PUSH \\
        LogotipoID & $\mathbb N$ & \xmark & ID del documento con el logotipo \\
        ApiKey & Alfanumérico & \cmark & Clave de seguridad de la API.\\
        PlayStoreUrl & URL & \xmark & Url de la aplicación en la PlayStore \\
        AppStoreUrl & URL & \xmark & Url de la aplicación en la AppStore. \\
        \hline
    \end{tabular}%}
    \caption{Atributos del tipo de entidad Aplicaciones}
    \label{cuadro:atributos-tipo-entidad-aplicaciones}
\end{table} 

\begin{table}[h!]
    \rowcolors{2}{gray!25}{white}
    \centering
    %\resizebox{\textwidth}{!}{%
    \begin{tabular}{|ll|}
        \hline
        \rowcolor[HTML]{9B9B9B} 
        \multicolumn{1}{|c}{\cellcolor[HTML]{9B9B9B}{\color[HTML]{FFFFFF} Atributo}} & \multicolumn{1}{c|}{\cellcolor[HTML]{9B9B9B}{\color[HTML]{FFFFFF} Valor}} \\ \hline
        AplicacionID & 1 \\
        Nombre & ``Escuela Politécnica Superior de Córdoba'' \\
        Versión & ``1.0.0.0'' \\
        Activo & Verdadero \\
        SubDominio & ``epsc'' \\
        CloudKey & <texto encriptado> \\
        Usuario & epspush \\
        Clave & <texto encriptado> \\
        LogotipoID & 8 \\
        ApiKey & `AIzaSyCqhjgrPTPSOFyLyos5gfN47TJ0HnNA\_LA' \\
        PlayStoreUrl & `https://play.google.com/\dots?id=com.pushnews.epsc' \\
        AppStoreUrl & `https://apps.apple.com/\dots/pushnews-epsc' \\
        \hline
    \end{tabular}%}
    \caption{Ejemplo de registro de tipo Aplicación}
    \label{cuadro:ejemplo-aplicacion}
\end{table}

\subsection{Usuarios}

\subsubsection*{Descripción}
Los usuarios del sistema tienen un email único y una clave para acceder. Además, se almacena su nombre y una marca para indicar si el email ha sido confirmado.

Los usuarios tienen un rol de editor o administrador. Un editor se vincula a una o varias aplicaciones en las que puede administrar las comunicaciones, categorías, etc. Los administradores gestionan las aplicaciones existentes, usuarios, parámetros del sistema\dots y también pueden actuar como editores dentro de cualquier aplicación.

\subsubsection*{Restricciones}
No habrá usuarios con el mismo email.

\subsubsection*{Características}
\begin{description}[nosep,style=multiline,labelindent=0.8cm,leftmargin=4cm,font=\normalfont]
    \item[Nombre] Usuarios
    \item[Id. principal] UsuarioID
    \item[Id. alternativo] Email
    \item[Atrib. heredados] RolID (Roles.RolID)
\end{description}

\subsubsection*{Atributos de la entidad}
En la tabla \ref{cuadro:atributos-tipo-entidad-usuarios} se describen todos los atributos de la entidad. Así mismo, en la tabla \ref{cuadro:ejemplo-usuario} se muestra un ejemplo de los valores que tendría un registro de usuario.

\begin{table}[h!]
    \rowcolors{2}{gray!25}{white}
    \centering
    %\resizebox{\textwidth}{!}{%
    \begin{tabular}{|llcp{7.5cm}|}
        \hline
        \rowcolor[HTML]{9B9B9B}
        \multicolumn{1}{|l}{\cellcolor[HTML]{9B9B9B}{\color[HTML]{FFFFFF} Atributo}} & 
        \multicolumn{1}{c}{\cellcolor[HTML]{9B9B9B}{\color[HTML]{FFFFFF} Dominio}} &
        \multicolumn{1}{c}{\cellcolor[HTML]{9B9B9B}{\color[HTML]{FFFFFF} Obl.}} &
        \multicolumn{1}{c|}{\cellcolor[HTML]{9B9B9B}{\color[HTML]{FFFFFF} Descripción}} \\
        UsuarioID & $\mathbb N$ & \cmark & Identificador de usuario \\
        Email & Alfanumérico & \cmark & Email del usuario \\
        Nombre & Alfanumérico & \cmark & Nombre del usuario \\
        Apellidos & Alfanumérico & \cmark & Apellidos del usuario \\
        Clave & Alfanumérico & \cmark & Clave de acceso del usuario \\
        Activo & Booleano & \cmark & Aplicación activa/inactiva \\
        EmailConfirmado & Booleano & \cmark & El usuario ha confirmado el email \\
        Creado & Fecha & \cmark & Fecha de creación del registro \\
        Actualizado & Fecha & \xmark & Fecha de actualización del registro \\
        RolID & $\mathbb N$ & \cmark & ID del rol del usuario\\
        \hline
    \end{tabular}%}
    \caption{Atributos del tipo de entidad Usuarios}
    \label{cuadro:atributos-tipo-entidad-usuarios}
\end{table}


\begin{table}[h!]
    \rowcolors{2}{gray!25}{white}
    \centering
    %\resizebox{\textwidth}{!}{%
    \begin{tabular}{|ll|}
        \hline
        \rowcolor[HTML]{9B9B9B} 
        \multicolumn{1}{|c}{\cellcolor[HTML]{9B9B9B}{\color[HTML]{FFFFFF} Atributo}} & \multicolumn{1}{c|}{\cellcolor[HTML]{9B9B9B}{\color[HTML]{FFFFFF} Valor}} \\ \hline
        UsuarioID & 1 \\
        Email & ``joselopez@mailsrv.com'' \\
        Nombre & ``José'' \\
        Apellidos & ``López Pérez'' \\
        Clave & <texto encriptado> \\
        Activo & Verdadero \\
        EmailConfirmado & Falso \\
        Creado & 2020-11-07 11:03:00 \\
        Actualizado & 2020-11-08 16:52:00 \\
        RolID & 1 \\
        \hline
    \end{tabular}%}
    \caption{Ejemplo de registro de tipo Usuario}
    \label{cuadro:ejemplo-usuario}
\end{table}

\subsection{Comunicaciones}

\subsubsection*{Descripción}
Una comunicación es una publicación realizada por un editor en una aplicación. La comunicación contiene datos sobre su contenido como título, cuerpo, categoría y recursos adjuntos como enlaces, documentos o imágenes, fecha de publicación, etc. Además, tiene otros datos como su estado (publicada, borrada, activa...), estado de notificaciones push (notificada o no, recordatorio...). Las comunicaciones podrán ser inmediatas si se publican inmediatamente o programadas, cuando se programan para ser publicadas en una fecha y hora concretas.

\subsubsection*{Restricciones}
No habrá usuarios con el mismo email.

\subsubsection*{Características}
\begin{description}[nosep,style=multiline,labelindent=0.8cm,leftmargin=4cm,font=\normalfont]
    \item[Nombre] Comunicaciones
    \item[Id. principal] ComunicacionID
    \item[Id. alternativo] Ninguno
    \item[Atrib. heredados] UsuarioID (Usuarios), CategoriaID (Categorias), ImagenDocu\linebreak mentoID (Documentos), AdjuntoDocumentoID (Documentos)
\end{description}

\subsubsection*{Atributos de la entidad}
En la tabla \ref{cuadro:atributos-tipo-entidad-comunicaciones} se describen todos los atributos de la entidad. Así mismo, en la tabla \ref{cuadro:ejemplo-comunicacion} se muestra un ejemplo de los valores que tendría un registro de característica.

\begin{table}[ht]
    \rowcolors{2}{gray!25}{white}
    \centering
    %\resizebox{\textwidth}{!}{%
    \begin{tabular}{|llcp{7.2cm}|}
        \hline
        \rowcolor[HTML]{9B9B9B}
        \multicolumn{1}{|l}{\cellcolor[HTML]{9B9B9B}{\color[HTML]{FFFFFF} Atributo}} & 
        \multicolumn{1}{c}{\cellcolor[HTML]{9B9B9B}{\color[HTML]{FFFFFF} Dominio}} &
        \multicolumn{1}{c}{\cellcolor[HTML]{9B9B9B}{\color[HTML]{FFFFFF} Obl.}} &
        \multicolumn{1}{c|}{\cellcolor[HTML]{9B9B9B}{\color[HTML]{FFFFFF} Descripción}} \\
        ComunicacionID & $\mathbb N$ & \cmark & Identificador de la comunicación \\
        UsuarioID & $\mathbb N$ & \cmark & Identificador del usuario editor \\
        CategoriaID & $\mathbb N$ & \cmark & Identificador de la categoría de la comunicación \\
        FechaCreacion & Fecha & \cmark & Fecha de creación \\
        FechaPublicacion & Fecha & \cmark & Fecha de publicación \\
        Activo & Booleano & \cmark & Comunicación activa/inactiva \\
        Borrado & Booleano & \cmark & Marca de borrado de la comunicación \\
        FechaBorrado & Fecha & \xmark & Fecha de borrado de la comunicación \\
        TimeStamp & $\mathbb N$ & \xmark & Marca de tiempo de la publicación \\
        PushEnviada & Booleano & \cmark & Notificación push enviada/no enviada \\
        PushFecha & Fecha & \xmark & Fecha de notificación push \\
        Titulo & Alfanumérico & \cmark & Título de la comunicación \\
        Descripcion & Booleano & \cmark & Cuerpo de la comunicación \\
        UltimaEdicionIP & Alfanumérico & \xmark & IP desde la que se realizó la última edición \\
        Autor & Alfanumérico & \cmark & Nombre del autor \\
        Destacado & Booleano & \cmark & Publicación destacada/no destacada \\
        RecordatorioTitulo & Alfanumérico & \xmark & Texto del recordatorio \\
        RecordatorioFecha & Fecha & \xmark & Fecha del mensaje recordatorio \\
        PushRecordatorio & Booleano & \xmark & Mensaje push del recordatorio enviado/no enviado \\
        Instantanea & Booleano & \cmark & Publicación instantánea/diferida \\
        ImagenDocumentoID & $\mathbb N$ & \xmark & ID de la imagen adjunta \\
        ImagenTitulo & Alfanumérico & \xmark & Título de imagen adjunta \\
        AdjuntoDocumentoID & $\mathbb N$ & \xmark & ID del documento adjunto \\
        AdjuntoTitulo & Alfanumérico & \xmark & Título de documento adjunto \\
        EnlaceTitulo & Alfanumérico & \xmark & ID del rol del usuario \\
        Enlace & URL & \xmark & Enlace a recurso adjunto \\
        YoutubeTitulo & Alfanumérico & \xmark & Título del vídeo de Youtube adjunto \\
        Youtube & URL & \xmark & URL del video de Youtube adjunto \\
        GeoPosicionTitulo & Alfanumérico & \xmark & Título de la localización adjunta \\
        GeoPosicionLatitud & $\mathbb R$ & \xmark & Latitud de la localización adjunta \\
        GeoPosicionLongitud & $\mathbb R$ & \xmark & Longitud de la localización adjunta \\
        GeoPosicionDireccion & Alfanumérico & \xmark & Dirección de la localización adjunta \\
        GeoPosicionLocalidad & Alfanumérico & \xmark & Localidad de la localización adjunta \\
        GeoPosicionProvincia & Alfanumérico &\xmark & Provincia de la localización adjunta \\
        GeoPosicionPais & Alfanumérico & \xmark & País de la localización adjunta \\
        \hline
    \end{tabular}%}
    \caption{Atributos del tipo de entidad Comunicaciones}
    \label{cuadro:atributos-tipo-entidad-comunicaciones}
\end{table}

\begin{table}[ht]
    \rowcolors{2}{gray!25}{white}
    \centering
    %\resizebox{\textwidth}{!}{%
    \begin{tabular}{|ll|}
        \hline
        \rowcolor[HTML]{9B9B9B} 
        \multicolumn{1}{|c}{\cellcolor[HTML]{9B9B9B}{\color[HTML]{FFFFFF} Atributo}} & \multicolumn{1}{c|}{\cellcolor[HTML]{9B9B9B}{\color[HTML]{FFFFFF} Valor}} \\ \hline
        ComunicacionID & 19547 \\
        UsuarioID & 549 \\
        CategoriaID & 17 \\
        FechaCreacion & 2020-11-08 15:48:14 \\
        FechaPublicacion & 2020-11-10 10:00:00 \\
        Activo & Verdadero \\
        Borrado & Falso \\
        FechaBorrado & Nulo \\
        TimeStamp & 1604832148295 \\
        PushEnviada & Verdadero \\
        PushFecha & 2020-11-10 10:00:00 \\
        Titulo & ``RENOVACIÓN DE MATRÍCULA (GRADO)'' \\
        Descripcion & ``Se gestiona por Sigma según calendario\dots'' \\
        UltimaEdicionIP & 127.0.0.1 \\
        Autor & ``Eduardo Arroyo Ramírez'' \\
        Destacado & Falso \\
        RecordatorioTitulo & Nulo \\
        RecordatorioFecha & Nulo \\
        PushRecordatorio & Nulo \\
        Instantanea & Falso \\
        ImagenDocumentoID & 4981 \\
        ImagenTitulo & ``Calendario de renovaciones'' \\
        AdjuntoDocumentoID & Nulo \\
        AdjuntoTitulo & Nulo \\
        EnlaceTitulo & ``Solicitar renovación'' \\
        Enlace & ``http://uco.es/eps/es/renovacion-de-matricula-grado'' \\
        YoutubeTitulo & Nulo \\
        Youtube & Nulo \\
        GeoPosicionTitulo & ``Secretaría EPS'' \\
        GeoPosicionLatitud & 37.913484 \\
        GeoPosicionLongitud & -4.721551 \\
        GeoPosicionDireccion & Aulario Averroes, Rabanales \\
        GeoPosicionLocalidad & Córdoba \\
        GeoPosicionProvincia & Córdoba \\
        GeoPosicionPais & España \\
        \hline
    \end{tabular}%}
    \caption{Atributos del tipo de entidad Comunicaciones}
    \label{cuadro:ejemplo-comunicacion}
\end{table}

\subsection{Categorías}

\subsubsection*{Descripción}
Las categorías se definen para cada aplicación y agrupan las comunicaciones por temática. Tienen un nombre y un icono que las representa. Se ordenan según el criterio de los editores, para lo que cuentan con un campo orden. Pueden estar activas o inactivas. Si una categoría está inactiva, las comunicaciones de esta categoría no serán visibles ni se podrán crear nuevas comunicaciones en esta categoría.

\subsubsection*{Restricciones}
Ninguna

\subsubsection*{Características}
\begin{description}[nosep,style=multiline,labelindent=0.8cm,leftmargin=4cm,font=\normalfont]
    \item[Nombre] Usuarios
    \item[Id. principal] CategoriaID
    \item[Id. alternativo] Ninguno
    \item[Atrib. heredados] AplicacionID (Aplicaciones)
\end{description}

\subsubsection*{Atributos de la entidad}
En la tabla \ref{cuadro:atributos-tipo-entidad-categorias} se describen todos los atributos de la entidad. Así mismo, en la tabla \ref{cuadro:ejemplo-categoria} se muestra un ejemplo de los valores que tendría un registro de categoría.

\begin{table}[h!]
    \rowcolors{2}{gray!25}{white}
    \centering
    %\resizebox{\textwidth}{!}{%
    \begin{tabular}{|llcp{7.5cm}|}
        \hline
        \rowcolor[HTML]{9B9B9B}
        \multicolumn{1}{|l}{\cellcolor[HTML]{9B9B9B}{\color[HTML]{FFFFFF} Atributo}} & 
        \multicolumn{1}{c}{\cellcolor[HTML]{9B9B9B}{\color[HTML]{FFFFFF} Dominio}} &
        \multicolumn{1}{c}{\cellcolor[HTML]{9B9B9B}{\color[HTML]{FFFFFF} Obl.}} &
        \multicolumn{1}{c|}{\cellcolor[HTML]{9B9B9B}{\color[HTML]{FFFFFF} Descripción}} \\
        CategoriaID & $\mathbb N$ & \cmark & Identificador de la categoría \\
        AplicaciónID & $\mathbb N$ & \cmark & Identificador de la aplicación de la categoría \\
        Nombre & Alfanumérico & \cmark & Nombre de la categoría \\
        Icono & Alfanumérico & \cmark & Icono (fontawesome) de la categoría \\
        Orden & $\mathbb N$ & \cmark & Orden de la categoría \\
        Activo & Booleano & \cmark & Comunicación activa/inactiva \\
        \hline
    \end{tabular}%}
    \caption{Atributos del tipo de entidad Categorías}
    \label{cuadro:atributos-tipo-entidad-categorias}
\end{table}

\begin{table}[h!]
    \rowcolors{2}{gray!25}{white}
    \centering
    %\resizebox{\textwidth}{!}{%
    \begin{tabular}{|ll|}
        \hline
        \rowcolor[HTML]{9B9B9B} 
        \multicolumn{1}{|c}{\cellcolor[HTML]{9B9B9B}{\color[HTML]{FFFFFF} Atributo}} & \multicolumn{1}{c|}{\cellcolor[HTML]{9B9B9B}{\color[HTML]{FFFFFF} Valor}} \\ \hline
        CategoriaID & 17 \\
        AplicacionID & 3 \\
        Nombre & ``Secretaría'' \\
        Icono & ``fa-leanpub'' \\
        Orden & 3 \\
        Activo & Verdadero \\
        \hline
    \end{tabular}
    \caption{Ejemplo de registro de tipo Categorías}
    \label{cuadro:ejemplo-categoria}
\end{table}

\subsection{Características de aplicaciones}

\subsubsection*{Descripción}
Las características son funcionalidades que se pueden habilitar o deshabilitar para cada aplicación. Ejemplos de características serían la posibilidad de incrustar un vídeo de Youtube en las comunicaciones o el acceso al módulo de directorio comercial.

\subsubsection*{Restricciones}
No habrá dos características con el mismo nombre.

\subsubsection*{Características}
\begin{description}[nosep,style=multiline,labelindent=0.8cm,leftmargin=4cm,font=\normalfont]
    \item[Nombre] AplicacionesCaracteristicas
    \item[Id. principal] AplicacionCaracteristicaID
    \item[Id. alternativo] Ninguno
    \item[Atrib. heredados] Ninguno
\end{description}

\subsubsection*{Atributos de la entidad}
En la tabla \ref{cuadro:atributos-tipo-entidad-caracteristicas} se describen todos los atributos de la entidad. Así mismo, en la tabla \ref{cuadro:ejemplo-caracteristica} se muestra un ejemplo de los valores que tendría un registro de característica.

\begin{table}[h!]
    \rowcolors{2}{gray!25}{white}
    \centering
    %\resizebox{\textwidth}{!}{%
    \begin{tabular}{|llcp{5.9cm}|}
        \hline
        \rowcolor[HTML]{9B9B9B}
        \multicolumn{1}{|l}{\cellcolor[HTML]{9B9B9B}{\color[HTML]{FFFFFF} Atributo}} & 
        \multicolumn{1}{c}{\cellcolor[HTML]{9B9B9B}{\color[HTML]{FFFFFF} Dominio}} &
        \multicolumn{1}{c}{\cellcolor[HTML]{9B9B9B}{\color[HTML]{FFFFFF} Obl.}} &
        \multicolumn{1}{c|}{\cellcolor[HTML]{9B9B9B}{\color[HTML]{FFFFFF} Descripción}} \\
        AplicacionCaracteristicaID & $\mathbb N$ & \cmark & Identificador de la característica \\
        Nombre & Alfanumérico & \cmark & Nombre de la característica \\
        Activo & Booleano & \cmark & Característica activa/inactiva \\
        \hline
    \end{tabular}%}
    \caption{Atributos del tipo de entidad Características}
    \label{cuadro:atributos-tipo-entidad-caracteristicas}
\end{table}

\begin{table}[h!]
    \rowcolors{2}{gray!25}{white}
    \centering
    %\resizebox{\textwidth}{!}{%
    \begin{tabular}{|ll|}
        \hline
        \rowcolor[HTML]{9B9B9B} 
        \multicolumn{1}{|c}{\cellcolor[HTML]{9B9B9B}{\color[HTML]{FFFFFF} Atributo}} & \multicolumn{1}{c|}{\cellcolor[HTML]{9B9B9B}{\color[HTML]{FFFFFF} Valor}} \\ \hline
        AplicacionCaracteristicaID & 4 \\
        Nombre & ``DirectorioComercial'' \\
        Activo & Verdadero \\
        \hline
    \end{tabular}
    \caption{Ejemplo de registro de tipo Características}
    \label{cuadro:ejemplo-caracteristica}
\end{table}

\subsection{Terminales}

\subsection{Accesos}


\section {Análisis de los tipos de interrelación}

\subsection{Aplicación-Característica}

\subsection{Aplicación-Usuario}

\subsection{Aplicación-Categoría}

\subsection{Aplicación-Terminal}

\subsection{Categoría-Comunicación}

\subsection{Comunicaciones-Accesos-Terminales}

\section{Modelo Entidad-Interrelación}

\chapter{Análisis funcional}
En este capítulo se detalla el funcionamiento del sistema. Primero se identificará a los actores que intervienen en el sistema, indicando los objetivos de cada uno. Después se presentarán los principales escenarios del sistema utilizando casos de uso.

%%%%%%%%%%%%%%%%%%%%%%%%%%%%%%%%%%%%%%%%%%%%%%%%%%%%%%%%%%%%%%%%%%%%%%%%%%%%%%%%%%%%%%%%%%%%%%%%%%%
%%%%%%%%%%%%%%%%%%%%%%%%%%%%%%%%%%%%%%%%%%%%%%%%%%%%%%%%%%%%%%%%%%%%%%%%%%%%%%%%%%%%%%%%%%%%%%%%%%%
%%%%%%%%%%%%%%%%%%%%%%%%%%%%%%%%%%%%%%%%%%%%%%%%%%%%%%%%%%%%%%%%%%%%%%%%%%%%%%%%%%%%%%%%%%%%%%%%%%%
%%%%%%%%%%%%%%%%%%%%%%%%%%%%%%%%%%%%%%%%%%%%%%%%%%%%%%%%%%%%%%%%%%%%%%%%%%%%%%%%%%%%%%%%%%%%%%%%%%%
%%%%%%%%%%%%%%%%%%%%%%%%%%%%%%%%%%%%%%%%%%%%%%%%%%%%%%%%%%%%%%%%%%%%%%%%%%%%%%%%%%%%%%%%%%%%%%%%%%%
\section{Identificación de actores}
Como ya se introdujo en \ref{objetivos-especificos-gestor-contenidos}, en \emph{PushNews} existen tres tipos de actores o tres perfiles. A continuación se profundizará en los objetivos de cada uno de ellos.

\subsection{Lector}
Cualquier usuario que, incluso sin estar previamente registrado y sin haberse identificado en el sistema, consulte un comunicado. Los principales objetivos de los lectores son:
\begin{enumerate}
    \item Visualizar los comunicados publicadas.
    \item Obtener los recursos adjuntos a los comunicados (imágenes, documentos\dots).
    \item Saber cuando un nuevo comunicado ha sido publicado.
\end{enumerate}

\subsection{Editor}
Generalmente, los editores pertenecerán a la organizaciones de los clientes y su misión es publicar y mantener los comunicados así como consultar las métricas. Sus objetivos más importantes son:
\begin{enumerate}
    \item Publicar comunicados
    \item Gestión o mantenimiento de los comunicados (modificación, eliminación\dots)
    \item Gestión de las categorías de comunicados de la aplicación.
    \item Gestión de otros datos de las aplicaciones como lugares de interés, teléfonos, etc.
    \item Ver las métricas de accesos a los comunicados
\end{enumerate}

\subsection{Administrador}
Los administradores se encargan del mantenimiento del sistema desde la organización de \emph{PushNews}. Entre sus objetivos principales encontramos:
\begin{enumerate}
    \item Mantenimiento general del servicio.
    \item Administrar las aplicaciones (altas, bajas, etc\dots).
    \item Administrar las características asociadas a las aplicaciones.
    \item Administración de los usuarios (editores) de las aplicaciones. 
    \item Realizar labores de editor de cualquier aplicación.
\end{enumerate}

%%%%%%%%%%%%%%%%%%%%%%%%%%%%%%%%%%%%%%%%%%%%%%%%%%%%%%%%%%%%%%%%%%%%%%%%%%%%%%%%%%%%%%%%%%%%%%%%%%%
%%%%%%%%%%%%%%%%%%%%%%%%%%%%%%%%%%%%%%%%%%%%%%%%%%%%%%%%%%%%%%%%%%%%%%%%%%%%%%%%%%%%%%%%%%%%%%%%%%%
%%%%%%%%%%%%%%%%%%%%%%%%%%%%%%%%%%%%%%%%%%%%%%%%%%%%%%%%%%%%%%%%%%%%%%%%%%%%%%%%%%%%%%%%%%%%%%%%%%%
%%%%%%%%%%%%%%%%%%%%%%%%%%%%%%%%%%%%%%%%%%%%%%%%%%%%%%%%%%%%%%%%%%%%%%%%%%%%%%%%%%%%%%%%%%%%%%%%%%%
%%%%%%%%%%%%%%%%%%%%%%%%%%%%%%%%%%%%%%%%%%%%%%%%%%%%%%%%%%%%%%%%%%%%%%%%%%%%%%%%%%%%%%%%%%%%%%%%%%%
\section{Análisis de casos de uso}
Larman define caso de uso como ``una colección de escenarios con éxito y fallo relacionados, que describe a los actores utilizando un sistema para satisfacer un objetivo.'' \cite{Larman2004}. En este apartado se presentarán algunos casos de uso del sistema que requieren una atención especial y se omitirán los que, por su simplicidad, quedan suficientemente descritos en el apartado \ref{requisitos-funcionales} de requisitos funcionales.

\subsection{Consulta de un comunicado}

\subsubsection*{Descripción}
Un lector consulta un comunicado. El sistema contabiliza la consulta para las estadísticas de acceso.

\subsubsection*{Actores}
\begin{itemize}
    \item Lector: quiere leer el contenido del comunicado.
    \item Sistema: tiene que registrar el acceso para que se refleje en las estadísticas.
\end{itemize}

\subsubsection*{Precondiciones}
\begin{itemize}
    \item El comunicado existe y está activo y publicado.
\end{itemize}

\subsubsection*{Descripción}

\begin{enumerate}
    \item El lector solicita la lectura de un comunicado.
    \item El sistema carga el comunicado.
    \item El sistema comprueba que el comunicado está activo y publicado.
    \item El sistema comprueba si el terminal del lector está dado de alta.
    \begin{enumerate}
        \item Si el terminal está dado de alta actualiza la fecha de último acceso.
        \item Si el terminal no está dado de alta, lo registra.
    \end{enumerate}
    \item El sistema registra el acceso del terminal al comunicado.
    \item El sistema muestra el contenido del comunicado al lector.
\end{enumerate}

\subsubsection*{Escenarios alternativos}
\begin{itemize}
    \item Si el comunicado no existe, no está activo o no está publicado, el lector verá un mensaje de error y este suceso quedará reflejado en el log.
    \item Si se produce un error al actualizar los datos del terminal o del acceso al comunicado, el lector verá el contenido del comunicado igualmente pero este suceso quedará reflejado en el log.
\end{itemize}

\subsubsection*{Postcondiciones}
\begin{itemize}
    \item El comunicado se muestra al lector.
    \item La visita queda contabilizada en el sistema.
\end{itemize}

%%%%%%%%%%%%%%%%%%%%%%%%%%%%%%%%%%%%%%%%%%%%%%%%%%%%%%%%%%%%%%%%%%%%%%%%%%%%%%%

\subsection{Publicación de un comunicado}

\subsubsection*{Definición}
Un editor escribe un comunicado en la web y el sistema lo registra y envía las notificaciones correspondientes.

\subsubsection*{Actores}
\begin{itemize}
    \item Editor: quiere publicar un comunicado
    \item Sistema: tiene que registrar el comunicado y enviar las notificaciones oportunas.
\end{itemize}

\subsubsection*{Precondiciones}
\begin{itemize}
    \item El editor está registrado el sistema y asociado a la aplicación en la que se publica el comunicado.
    \item La aplicación está activa.
\end{itemize}

\subsubsection*{Descripción}

\begin{enumerate}
    \item El editor crea un comunicado inmediato (con fecha-hora de publicación igual o anterior a la fecha y hora actuales).
    \item El sistema registra el comunicado.
    \item El sistema comprueba la fecha de publicación.
    \begin{enumerate}
        \item Si la fecha es anterior o igual a la actual (publicación instantánea) se envía el mensaje push y se marca el comunicado con el flag de push enviada y la fecha de envío.
        \item Si la publicación no es inmediata
    \end{enumerate}
    \item El sistema envía la notificación push.
    \item El sistema actualiza el comunicado con la fecha de envío de la notificación push y la marca de push enviada.
\end{enumerate}

\subsubsection*{Escenarios alternativos}
\begin{itemize}
    \item Si el editor ha añadido archivo adjunto o imagen, estos se cargan, se almacenan en disco y se registran en la tabla de Documentos para posteriormente ser vinculados al comunicado al ser creado.
    \item Si el envío de la notificación push falla, no se actualizan la marca ni la fecha correspondientes.
\end{itemize}

\subsubsection*{Postcondiciones}
\begin{itemize}
    \item El comunicado queda registrado y aparece en la rejilla de comunicados en la web.
\end{itemize}

%%%%%%%%%%%%%%%%%%%%%%%%%%%%%%%%%%%%%%%%%%%%%%%%%%%%%%%%%%%%%%%%%%%%%%%%%%%%%%%

\subsection{Envío de notificaciones push programadas}

\subsubsection*{Definición}
El sistema comprueba periódicamente las las notificaciones push de publicación o recordatorio que estén pendientes de enviar y las envía.

Un comunicado tiene notificación de publicación pendiente de envío cuando, estando activo, no borrado y con fecha de publicación anterior a la actual, no está marcado como ``notificado'' (atributo PushEnviado). Esto puede darse en dos circunstancias: cuando al publicar el comunicado, el envío de la notificación push falla y cuando el comunicado tiene fecha de publicación futura (no inmediata) y no se envía la notificación justo al ser publicado. Se considera que un comunicado tiene notificación de recordatorio pendiente de envío si está activo, no borrado, con fecha de recordatorio no nula y anterior a la actual y no está marcado como ``recordatorio enviado'' (atributo PushRecordatorio).

\subsubsection*{Actores}
\begin{itemize}
    \item Sistema: Envía las notificaciones push de publicación o recordatorio pendientes de los comunicados.
\end{itemize}

\subsubsection*{Precondiciones}
Ninguna

\subsubsection*{Descripción}

\begin{enumerate}
    \item El sistema recorre los comunicados con notificaciones pendientes de envío ordenados por aplicación.
    \item Para cada comunicado con notificación de publicación o recordatorio pendiente:
    \begin{itemize}
        \item Se envía la notificación push que corresponda (de publicación o recordatorio) utilizando las credenciales y el canal de la aplicación.
        \item Se actualiza el comunicado con la fecha de envío de la notificación push y la marca de push enviada.
    \end{itemize}
\end{enumerate}

\subsubsection*{Escenarios alternativos}
\begin{itemize}
    \item Si falla el envío de una notificación push, no se actualizará el comunicado correspondiente y se procesará de nuevo en el siguiente intento.
\end{itemize}

\subsubsection*{Postcondiciones}
\begin{itemize}
    \item El comunicado refleja que el envío de la notificación push se ha realizado y la fecha y hora del envío.
\end{itemize}
\part{Diseño}
\chapter{Diseño de datos}
En este apartado se obtendrá el modelo relacional a partir de los diagramas de la base de datos del apartado \ref{analisis-tipos-entidad}. El modelo relacional se compone de los elementos que se indican a continuación:

\subsubsection*{Entidad}
Cada una de las tablas que componen el modelo.

\subsubsection*{Atributo} Es una columna de una tabla. Cada una tiene un nombre y un dominio asociados.

\subsubsection*{Dominio}
Es el conjunto de valores que puede tomar un campo.

\subsubsection*{Tupla}
Es un conjunto formado por un valor de cada atributo de la tabla. No pueden existir dos tuplas iguales en una tabla.

\subsubsection*{Clave}
Es el atributo o conjunto de atributos de una tabla por cuyos valores identifica unívocamente a cada tupla.

Para generar el modelo relacional a partir del esquema entidad-interrelación se utilizan las reglas de normalización y transformación que se indican en \textit{Bases de datos: desde Chen hasta Codd con Oracle} \cite{basesdedatos-rama} que evitan redundancias y problemas de integridad en los datos.

Para facilitar la posterior implementación de la capa de negocio y simplificar la referencia a registros en la solicitudes HTTP se ha tomado la decisión de crear claves auto-numéricas en la mayoría de tablas.

En la figura \ref{fig:esquema-relacional} se representan todas las tablas, relaciones, atributos y claves del esquema relacional de la base de datos de \emph{PushNews}.

\begin{figure}[ht]
    \centering
    \includegraphics[scale=.59]{esquema_base_datos}
    \caption{Esquema relacional}
    \label{fig:esquema-relacional}
\end{figure}

\chapter{Diseño de clases}
\chapter{Arquitectura}

\section{Patrón MVC}

Modelo-vista-controlador (MVC) es un patrón de arquitectura de software que separa los datos y principalmente lo que es la lógica de negocio de una aplicación de su representación y el módulo encargado de gestionar los eventos y las comunicaciones. Para ello MVC propone la construcción de tres componentes distintos que son el modelo, la vista y el controlador, es decir, por un lado define componentes para la representación de la información, y por otro lado para la interacción del usuario. Este patrón de arquitectura de software se basa en las ideas de reutilización de código y la separación de conceptos, características que buscan facilitar la tarea de desarrollo de aplicaciones y su posterior mantenimiento \cite{wiki-mvc}.

\begin{figure}[h]
    \centering
    \begin{tikzpicture}[node distance=1cm, auto]
        \tikzset{
            mynode/.style={rectangle,rounded corners,draw=black, top color=white, bottom color=yellow!50,very thick, inner sep=1em, minimum size=3em, text centered},
            myarrow/.style={->, >=latex', shorten >=1pt, thick},
            mylabel/.style={text width=7em, text centered} 
        }
        \node[mynode] (controlador) {Controlador};
        \node[below=3cm of controlador] (dummy) {}; 
        \node[mynode, left=of dummy] (vista) {Vista};
        \node[mynode, right=of dummy] (modelo) {Modelo};
        
        \draw[myarrow] (controlador.south) -- (modelo.north);	
        \draw[myarrow] (vista.north) -- (controlador.south);
        \draw[myarrow] (modelo.west) -- (vista.east);
    \end{tikzpicture}
    \medskip
    \caption{Esquema de la arquitectura Modelo-Vista-Controlador}
\end{figure}

\section{Artefactos}

\begin{figure}[ht]
    \centering
    \begin{tikzpicture}[every fit/.style={inner sep=0pt, outer sep=0pt, draw}]
      \begin{scope}
        \node [fit={( 0,0) (12,1)}, label=center:{Acceso a datos}] {};
        \node [fit={( 0,1) (12,2)}, label=center:{Biblioteca de Negocio}] {};
        \node [fit={( 0,2) ( 4,3)}, label=center:{Aplicación WEB}] {};
        \node [fit={( 4,2) ( 9,3)}, label=center:{Servicio WEB}] {};
        \node [fit={( 9,2) (12,3)}, label=center:{Pr. Auxiliares}] {};
        \node [fit={( 4,3) ( 6,4)}, label=center:{App 1}] {};
        \node [fit={( 6,3) ( 7,4)}, label=center:{\dots}] {};
        \node [fit={( 7,3) ( 9,4)}, label=center:{App N}] {};
      \end{scope}
    \end{tikzpicture}
  \end{figure}
\chapter{Bibliotecas de negocio}

Soportan la lógica de negocio, el esquema de autenticación y autorización y el acceso a datos. Enfoque por tipo de entidad.

\section{Biblioteca de persistencia}


\section{Operaciones de negocio}
\chapter{Aplicación WEB}
La parte WEB de este proyecto  --incluído el servicio para las apps-- estará construida con .NET MVC, un framework de Microsoft para el desarrollo de aplicaciones WEB que implementa el patrón de diseño modelo-vista-controlador. .NET MVC es software libre y se distribuye bajo licencia Apache 2.0.

\section{Áreas}

\section{Filtros de acción}

\section{Controladores}

\section{Modelos}

\section{Vistas}

\section{Servicios}
\subsection{Seguridad y autenticación}
Windows Identity Foundation
\subsection{Servicio de correo electrónico}
\subsection{Mensajería push}
\include{capitulos/306-servicio-web}
\include{capitulos/307-programas-auxiliares}
\include{capitulos/308-aplicacion-movil-tipo}
\chapter{Diseño de la interfaz}

\section{Interfaz WEB}
En este apartado se describirá, la arquitectura de la solución, cada una de las partes de las que consta y la interfaz de usuario.


\section{Interfaz de la aplicación móvil tipo}
\part{Pruebas}
\part{Conclusiones y futuras mejoras}
\include{capitulos/501-conclusiones}
\chapter{Futuras mejoras}

\section{Capa de servicios}
Exponer clases POCO a modo de vistas en lugar de exponer las entidades. Es decir, PersonasServicio.BuscarPersona() debe devolver PersonaModel y no PersonaEntity.

\section{Aplicación web}

\subsection{Backend}
Inyección de dependencias
Prescindir de OWIN

\subsection{Frontend}
Sustituir Telerik por otro Framework

\subsection{Nuevos módulos de características}

\subsection{Explorador de documentos de la aplicación}
Componente para gestionar y reutilizar documentos asociados a la aplicación.

\subsection{Informes}
Generación de informes a partir de los datos de impacto de los comunicados.
\cleardoublepage

% BIBLIOGRAFÍA -----------------------------------------------------------------
\bibliographystyle{ieeetr}
\renewcommand{\refname}{Bibliografía}
\bibliography{00-memoria-tecnica}
\addcontentsline{toc}{part}{Bibliografía}
\clearpage
% ------------------------------------------------------------------------------

\end{document}