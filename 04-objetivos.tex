\chapter{Objetivos}
El objetivo general de este trabajo es la creación de una plataforma para la difusión de comunicados por parte de organizaciones independientes que facilite la medida de su impacto.

Para la consecución de dicho objetivo, se establecen los siguientes objetivos específicos:

\section{Plataforma de publicación y seguimiento de comunicados}
Es necesario dotar al sistema de una herramienta que permita a los editores elaborar, programar, publicar y consultar las métricas de impacto de los comunicados. Para ello se desarrollará una aplicación web con acceso individualizado que cubrirá las siguientes necesidades según el tipo de usuario:
\begin{itemize}
    \item Consumidores (acceso público)
    \begin{itemize}
        \item Consulta pública de comunicados de la organización.
    \end{itemize}
    \item Editores (acceso restringido por organización)
    \begin{itemize}
        \item Administración de comunicados.
        \item Consulta de métricas de impacto de comunicaciones.
        \item Administración de tipos de registro independientes como categorías de comunicaciones, teléfonos, ubicaciones, etc. de la organización.
    \end{itemize}
    \item Administradores (responsables del mantenimiento del servicio)
    \begin{itemize}
        \item Administración de organizaciones.
        \item Administración de funcionalidades disponibles para cada organización.
        \item Administración de usuarios.
        \item Configuración del servicio.
    \end{itemize}
    \item Organizaciones
    \begin{itemize}
        \item Proporcionar una API pública para la consulta de comunicados mediante una herramienta de terceros.
    \end{itemize}
\end{itemize}

\section{Aplicación móvil tipo}
El sistema contará con una herramienta que permita a los consumidores acceder a los comunicados de una organización para consultarlos desde su teléfono móvil. Para ello se elaborará una aplicación móvil tipo. Esta aplicación tipo será precursora de las diferentes aplicaciones cliente, una por cada organización registrada en el sistema.

Además de a través de la aplicación móvil, los usuarios consumidores podrán consultar los comunicados a través de la parte publica del apartado web de su organización.