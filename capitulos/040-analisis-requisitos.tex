\chapter{Análisis de requisitos}
Este apartado se centra en la investigación del problema y los requisitos. Aquí se definirán los conceptos en el dominio del problema.

\section{Modelo de casos de uso}

¿Qué es un modelo de casos de uso?
Formato utilizado
Listado de casos de uso formato breve con identificador único
Con el fin de identificar los actores, conceptos y actividades del sistema se elaborará un \emph{modelo de casos de uso} \cite{Jacobson1999} \cite{Larman2004}

\begin{itemize}
    \item Publicar un comunicado
    \item Programar un comunicado
    \item Consultar la lista de comunicados
    \item Consultar la lista de comunicados por categoría
    \item Consultar un detalle de comunicado
    \item zxcv
\end{itemize}


\section{Modelo de dominio}

\begin{figure}
    \centering
    \begin{tikzpicture}[show background grid]
        \begin{class}[text width=6cm]{Class}{0 ,0}
            \attribute{+Public }
            \attribute{\#Protected }
            \attribute{-Private }
            \attribute{$\sim$ Package }
        \end{class}
        \begin{class}[text width=7cm]{BankAccount}{7,0}
            \attribute{+owner:String}
            \attribute{+balance:Dollar s}
            \operation{+deposit(amount:Dollars)}
            \operation{+withdrawal(amount:Dollars)}
            \operation{\#updateBalance(newBalance:Dollars)}
        \end{class}
    \end{tikzpicture}
    \caption{Modelo de dominio}
\end{figure}

\begin{figure}
    \begin{sequencediagram}
        \newthread{t}{:Thread}
        \newinst{a}{:A}
        \newinst{b}{:B}
        \begin{call}{t}{funcA()}{a}{return}
            \begin{call}{a}{funcA()}{b}{return}
            \end{call}
        \end{call}
    \end{sequencediagram}
    \caption{Modelo de interacción}
\end{figure}