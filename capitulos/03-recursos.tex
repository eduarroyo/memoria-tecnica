\chapter{Recursos}
En esta sección se clasificarán y detallarán los recursos necesarios para la realización del trabajo.

\section{Humanos}
En este apartado se identifica a las personas involucradas en el trabajo.
\begin{itemize}
    \item Prof. D. \emph{Domingo Ortiz Boyer} profesor del Departamento de Informática y Análisis Numérico de la Escuela Politécnica Superior de la Universidad de Córdoba, como tutor del proyecto.
    \item D. \emph{Eduardo Arroyo Ramírez}, encargado de realizar dicho proyecto siguiendo las directrices definidas por el tutor del mismo.
\end{itemize}

\section{Hardware}
Los recursos hardware son las herramientas físicas empleadas para el desarrollo y la puesta en marcha del proyecto.
\begin{itemize}
    \item Ordenador PC Compatible con procesador Intel i7 y 8GB de memoria RAM.
    \item Teléfono móvil Android.
    \item Conexión a Internet de banda ancha.
\end{itemize}

\section{Software}
Bajo esta denominación se engloban las herramientas software como el sistema operativo, los entornos de programación, bibliotecas, herramientas de documentación, etc.
\begin{itemize}
    \item S. O. Windows 10 Home Premium
    \item Microsoft Visual Studio Community 2019
    \item Telerik UI para ASP.NET MVC
    \item Microsoft SQL Server Express 2016
    \item Titanium Studio
    \item Internet Information Server 7
    \item Navegadores:
    \begin{itemize}
        \item Google Chrome
        \item Mozilla Firefox
        \item Microsoft Edge
    \end{itemize}
    \item Microsoft Visual Studido Code
    \item Paquete \LaTeX de edición de textos
\end{itemize}